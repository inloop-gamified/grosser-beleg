
\chapter{Konzept}\label{ch:konzept}

In diesem Kapitel soll ein Konzept erstellt werden, welches der Anforderungsspezifikation aus \Cref{ch:requirements} gerecht wird und Nutzer motiviert, die Codequalität eingereichter Lösungen zu verbessern. Dazu wird nachfolgend zunächst betrachtet, wie die statische Codeanalyse architekturell in INLOOP eingegliedert werden kann, um als Grundlage für die Detektion und Kommunikation von Code Smells zu dienen. Danach wird ein Game-Design entwickelt, welches Gamification-Elemente gezielt auswählt und auf den Zielkontext anwendet, um zur Verbesserung der Codequalität zu motivieren. Auf Grundlage dieses Game-Designs wird anschließend diskutiert, welche strukturellen und funktionellen Änderungen in INLOOP für die Umsetzung eines solchen Game-Designs notwendig sind und schließlich eine konkrete Modellstruktur konzipiert, auf deren Basis die Umsetzung der Gamification-Erweiterung durchgeführt werden kann.

\section{Codeanalyse}

\subsection{Regelwerk für Codierungsrichtlinien}

Grundlage der Codeanalyse soll das in \Cref{sec:qualityreview} beschriebene, im akademischen Kontext entwickelte QualityReview Framework von Dietz et al. \cite{dietz_teaching_2018} sein. Im Folgenden soll beschrieben werden, wie das Framework für diese Arbeit wiederverwendet werden kann.

\begin{figure}[H]
\centering
\includegraphics[width=\linewidth, bb=0 0 547 217]{qualityreview.pdf}
\caption{Architektur des QualityReview Frameworks von Dietz et al. \cite{dietz_teaching_2018}.}\label{fig:qualityreview}
\end{figure}

\noindent In \Cref{fig:qualityreview} ist die Architektur des Analyseframeworks gezeigt. Um eine beliebige Java-Codebasis (in \Cref{fig:qualityreview} illustriert durch Foo.java und Bar.java) mithilfe des Frameworks zu analysieren, werden die darin enthaltenen Java-Dateien in einen Gradle\footnote{Gradle. \url{https://gradle.org/} (Abgerufen am 11.8.2020)} Buildprozess gegeben. Für jedes der drei unterstützten Codeanalysetools (PMD, FindBugs/SpotBugs, Checkstyle) wird hierbei eine Spezifikation der Eingaben und der Ausgaben vorgenommen. Die eingegebenen Java-Dateien werden entsprechend der Spezifikation vom jeweiligen Codeanalysetool analysiert und das Analyseergebnis schließlich in einer standardisierten Form wie zum Beispiel XML ausgegeben.

Teil der Spezifikation ist das Regelwerk, nach dem die Codeanalysetools Code Smells detektieren. Das Eingabeformat des Regelwerks unterscheidet sich hierbei je nach Codeanalysetool (siehe \Cref{fig:qualityreview}). Checkstyle und PMD werden mit einem eigenen XML-Regelwerk konfiguriert, während FindBugs/SpotBugs lediglich mit Prüfparametern wie dem \enquote{reportLevel} konfiguriert wird. Diese Konfigurationen sind das Kernelement des Frameworks und sollen im Folgenden so adaptiert werden, dass diese in INLOOP integriert werden können und die in \Cref{sec:spezifikation} erläuterten Anforderungen erfüllen.

\subsection{Integrationskonzept der statischen Codeanalyse}

Die Prüfung auf funktionelle Korrektheit von Lösungen in INLOOP über Unit Tests wird durch eine besondere Komponente durchgeführt, welche sich \textit{TestRunner} nennt. Wird eine Lösung in INLOOP eingereicht, so initialisiert dieser TestRunner eine virtualisierte Laufzeitumgebung, in der die Dateien der Lösung dynamisch durch jUnit\footnote{jUnit 5. \url{https://junit.org/junit5/} (Abgerufen am 11.8.2020)} Tests getestet werden. Hierzu wird innerhalb eines individuellen Docker Containers ein Ant\footnote{Ant. \url{https://ant.apache.org/} (Abgerufen am 11.8.2020)} Buildprozess durchgeführt, welcher die hierzu notwendigen Schritte und die zu inkludierenden Dateien spezifiziert. Die Ant Buildkonfiguration wird hierbei zusammen mit der Aufgabe über Continuous Publishing \cite{morgenstern_continuous_2018} dynamisch von einem externen Git-Repository bezogen. Die Ausgabe der jUnit Testsuite wird wiederum im XML-Format gespeichert und schließlich vom TestRunner geparst, so dass die in der virtualisierten Laufzeitumgebung temporär abgelegten Ausgabedateien schließlich in der Datenbank von INLOOP persistiert werden.

Da sowohl die dynamische Codeanalyse, als auch die statische Codeanalyse relativ (im Vergleich zur Bereitstellung anderer Funktionalitäten in INLOOP) rechenintensiv sind, sollten diese nicht blockierend ausgeführt werden. Genauer sollen diese Prozesse vom normalen Request-Response-Prozess der Webanwendung entkoppelt werden. Ein weiteres Problem ist die Ausführung von hochgeladenem, potenziell schadhaftem Code. Der TestRunner bietet für beide Probleme eine hinreichend gute Lösung durch die Virtualisierung des Betriebssystems über Docker Container und die entkoppelte Ausführung dieser über eigene Service-Worker-Prozesse. Daher soll im Folgenden gezeigt werden, wie die statische Codeanalyse zusätzlich zur dynamischen Codeanalyse im TestRunner konzeptuell integriert werden kann, um schließlich zu zeigen, wie hierbei das QualityReview Framework eingesetzt werden kann.

\begin{figure}[H]
\centering
\includegraphics[width=\linewidth, bb=0 0 432 196]{integration.pdf}
\caption{Integrationskonzept des QualityReview Frameworks für die TestRunner Komponente von INLOOP.}\label{fig:integration}
\end{figure}

\noindent Das in \Cref{fig:integration} gezeigte Integrationskonzept ermöglicht die nahtlose Ergänzung der statischen Codeanalyse auf Grundlage des QualityReview Frameworks. Im Zentrum steht die bereits beschriebene TestRunner Komponente. Der durch diese gesteuerte Ant Buildprozess wird durch die Ausführung der Codeanalysetools PMD, Checkstyle und FindBugs/SpotBugs erweitert, wobei die Konfigurationen aus dem QualityReview Framework wiederverwendet werden. Die Spezifikation des Gradle Builds aus QualityReview wird hierzu auf eine Ant Spezifikation überführt. Nach der Ausführung des Ant Buildprozesses stehen die Ausgabedateien der statischen Codeanalysetools analog zum Gradle Buildprozess in standardisierter Form (XML) auf dem Dateisystem des virtualisierten Docker Containers zur Verfügung und können von hieraus für die weitere Verarbeitung verwendet werden.

\subsection{Semantische Code-Smell-Extraktion}\label{sec:code-smell-extraktion}

Mithilfe der im vorigen Abschnitt vorgestellten Architektur können Lösungen in INLOOP auf Verletzungen von Codierungsrichtlinien (Detektionen) statisch geprüft werden. Die Detektionen der Codeanalysetools sind dabei mehr oder weniger deskriptiv, beispielsweise beinhalten die Detektionen aus dem Checkstyle Tool (in der Version 7.6.1 aus dem QualityReview Framework) oder dem PMD Tool meist nur eine sehr kurze, prägnante Beschreibung des eigentlichen Fehlers und nie eine Beschreibung der zugrundeliegenden Maximen. Ein Studierender soll später jedoch nicht nur eine sehr kurze und potenziell missverständliche Meldung über die Detektion sehen (analog zur Anforderungsbeschreibung aus \Cref{sec:spezifikation}), sondern eine Erklärung erhalten, weshalb die zugrundeliegende, verletzte Codierungsrichtlinie möglicherweise eine Degradation der Codequalität zur Folge hat. Zur Illustration dessen soll folgender Java-Codeabschnitt aus einer hypothetischen Datei \enquote{Example.java} dienen.

\vspace{\abovedisplayskip}

\begin{lstlisting}[language = Java, firstnumber = 49]
// ...
public void do_something() {
    if (conditionIsTrue)
        callFunction();
}
// ...
\end{lstlisting}

\noindent Die Analyse durch Checkstyle mit der Checkstyle-Konfiguration aus dem QualityReview Framework detektiert hierbei unter anderem einen Verstoß gegen Codierungsrichtlinien in Zeile 51 und gibt die folgende XML-Datei aus.

\vspace{\abovedisplayskip}

\begin{lstlisting}[language = XML]
<?xml version="1.0" encoding="UTF-8"?>
<checkstyle version="7.6.1">
    <file name="/checker/input/Example.java">
        <error line="51"
               severity="error"
               message="if construct must use {}s."
               source="[URI].NeedBracesCheck"/>
    </file>
</checkstyle>
\end{lstlisting}

\noindent Bei der Analyse durch PMD wiederum wird unter anderem in Zeile 50 ein Verstoß gegen die Benennungskonventionen detektiert.

\vspace{\abovedisplayskip}

\begin{lstlisting}[language = XML]
<?xml version="1.0" encoding="UTF-8"?>
<pmd xmlns="http://pmd.sourceforge.net/report/2.0.0"
    xmlns:xsi="http://www.w3.org/2001/XMLSchema-instance"
    xsi:schemaLocation="[URL]/report_2_0_0.xsd"
    version="6.2.0" timestamp="2020-08-12T09:23:06.378">
    <file name="Example.java">
        <violation beginline="51" endline="51"
                   begincolumn="12" endcolumn="24"
                   rule="MethodNamingConventions" ruleset="Code Style"
                   class="Example" method="do_something"
                   externalInfoUrl="[URL]#methodnamingconventions"
                   priority="1">
            Method names should not contain underscores
        </violation>
    ...
    </file>
</pmd>
\end{lstlisting}

\noindent Zu sehen ist, dass die Nachrichten der Detektionen zwar jeweils eine kurze Beschreibung des Code Smells geben, jedoch nicht, aus welchem Grund dies die Codequalität potenziell mindert. Im illustrierten Beispiel müssten bei einer Ergänzung einer Instruktion in Zeile 53 geschweifte Klammern ergänzt werden, um diese Instruktion zusammen mit der Instruktion in 52 konditionell auszuführen. Wird dies vergessen, könnte dies zu Fehlern aufgrund einer nichtkonditionellen Ausführung der Instruktion kommen. Außerdem sollten Methoden in \enquote{lowerCamelCase} statt, wie in Zeile 50 in \enquote{snake\_case}, geschrieben werden, um zu den Konventionen der Programmiersprache konsistent zu bleiben und die Lesbarkeit zu verbessern. Solche semantische Beschreibungen der Detektionen soll der Studierende erhalten, um einerseits die Verständlichkeit zu verbessern, aber auch, um die zugrundeliegenden Maximen von \enquote{Clean Code} zu lehren. Dies dient als Mensch-Computer-Schnittstelle für die weitere Kommunikation der Code Smells, um schließlich die Codequalität eingereichter Lösungen nachhaltig verbessern zu können.

\begin{figure}[H]
\centering
\includegraphics[width=\linewidth, bb=0 0 547 239]{integration2.pdf}
\caption{Eine Architektur zur vereinheitlichten, semantischen Code-Smell-Extraktion auf Grundlage der Ausgaben von Codeanalysetools.}\label{fig:integration2}
\end{figure}

\noindent Die Codeanalysetools PMD, Checkstyle und FindBugs/SpotBugs inkludieren in deren standardisierten Ausgaben jeweils pro Detektion einen Identifikator für die Art des ausgelösten Detektionsmechanismus oder der zugrundeliegenden Codierungsrichtlinie. Auf Grundlage dessen ermöglicht die in \Cref{fig:integration2} gezeigte Architektur die Erweiterung der Detektionen um Erklärungen zu den Maximen. Da die Ausgabegrammatiken der Codeanalysetools stark voneinander abweichen, werden die detektierten Code Smells zunächst über einen jeweiligen, spezialisierten Parser in eine in der Datenbank von INLOOP persistierbare Form (ein \enquote{Set} von \enquote{Violations}) überführt. Hierzu wird die Wichtigkeit der Detektion in die drei Kategorien \enquote{information} für rein informative Detektionen, \enquote{warning} für mittelwichtige Detektionen und \enquote{error} für die wichtigsten Detektionen aufgeteilt. Die entsprechenden Parser bilden die Formate der Codeanalysetools hierauf ab. Beispielsweise werden die PMD-Prioritäten 1 und 2 auf die Wichtigkeit \enquote{error}, 3 und 4 auf \enquote{warning} und 5 auf \enquote{information} entsprechend zur Dokumentation der Prioritäten\footnote{PMD. \url{https://pmd.github.io/latest/pmd_userdocs_extending_rule_guidelines.html} (Abgerufen am 11.8.2020)} abgebildet. Die Detektionen werden schließlich ausgestattet mit einer jeweiligen textuellen Erklärung aus einer regelbasierten Erklärungsdatenbank, welche im Folgenden näher beschrieben werden soll.

\subsection{Regelbasierte Erklärungsdatenbank}

Um eine \enquote{Violation}, genauer die vereinheitlichte und persistierbare Abbildung der Codeanalysetool-Ausgaben, mit diesen textuellen Erklärungen zu ergänzen, werden die Identifikatoren (URIs) der Detektionen auf einen Identifikator der Erklärungsdatenbank und die dazugehörige textuelle Erklärung abgebildet.

\begin{figure}[H]
\centering
\includegraphics[width=\linewidth, bb=0 0 583 304]{edb.pdf}
\caption{Beispielabbildung für die Adaption der Identifikatoren von Detektionen auf Codeanalysetools zur Ergänzung von Code-Smell-spezifischen Erklärungen.}\label{fig:edb}
\end{figure}

\noindent \Cref{fig:edb} zeigt die Adaption der Identifikatoren, um die Detektionen mit in der Datenbank hinterlegten Erklärungen zu ergänzen. Zu sehen ist, dass die Erklärungsdatenbank zwischen drei Kategorien unterscheidet, den strukturellen, stilistischen und generellen Verstößen. Prinzipiell können diese Kategorien bei der Umsetzung dieses Konzepts frei gewählt werden, beispielsweise nach einer anderen Unterteilung, zum Beispiel in \enquote{Bad Practice}, \enquote{Correctness}, \enquote{Vulnerability} oder \enquote{Performance}, analog zu den Detektionskategorien in FindBugs\footnote{FindBugs. \url{http://findbugs.sourceforge.net/bugDescriptions.html} (Abgerufen am 12.8.2020)}/SpotBugs\footnote{SpotBugs. \url{https://spotbugs.readthedocs.io/en/latest/bugDescriptions.html} (Abgerufen am 14.9.2020)}. Die in \Cref{fig:edb} gezeigte Kategorisierung hängt zusammen mit der späteren Aufbereitung in Form eines Narratives, welches in \Cref{sec:narrativ-konzept} gezeigt wird. Außerdem frei wählbar sind die Identifikatoren der Erklärungen aus der Erklärungsdatenbank und damit die Granularität der Abbildung. Da die Abbildung zwischen den Identifikatoren der Codeanalysetools und den Identifikatoren der Erklärungen von der Parser-Infrastruktur (siehe \Cref{fig:integration2}) realisiert wird, können gemeinsame Schnittmengen im Regelsatz gebildet werden, die jeweils auf dieselbe Erklärung abbilden. Illustriert wird dies exemplarisch in \Cref{fig:edb} durch den PMD-Identifikator \enquote{MethodNamingConventions} und den FindBugs-Identifikator \enquote{NM\_METHOD\_NAMING\_CONVENTIONS}, welche beide auf die Erklärung hinter dem Erklärungsdatenbank-Identifikator \enquote{MethodNaming} abgebildet werden können, da den beiden Detektionsmechanismen dieselbe Maxime zugrunde liegt. Außerdem können in der Erklärungsdatenbank verhältnismäßig wenige oder viele Erklärungen hinterlegt sein, auf welche die Identifikatoren der Codeanalysetools abgebildet werden. Direkt abhängig davon ist jedoch die Spezifität der Erklärungen. Ein Nachteil des Konzeptes der Erklärungsdatenbank ist die Notwendigkeit der manuellen Erstellung der Abbildung in der Parser-Infrastruktur sowie die manuelle Erstellung der textuellen Erklärungen. Auf eine mögliche Strategie in der konkreten Implementation wird in \Cref{ch:implementation} eingegangen.


\section{Game-Design}\label{sec:game-design}

Auf Grundlage der in den vorigen Sektionen vorgestellten Konzepte zur Integration der statischen Codeanalyse in INLOOP kann nun ein Gamification-Konzept erstellt werden, welches dazu dienen soll, die semantischen Detektionen weiter aufzubereiten und den Nutzer dazu zu motivieren, die Detektionen zu lesen, zu verstehen und schließlich selbst umzusetzen. Gemäß dem in \Cref{sec:gamification-frameworks} vorgestellten Octalysis-Ansatz soll dafür im Folgenden ein nutzerorientiertes Game-Design erstellt werden. Hierzu wird zunächst eruiert, welche Bedürfnisse und Ziele des Nutzers motiviert werden sollen (Leitmotive). Anschließend wird hierum ein Narrativ entwickelt, um hierzu schließlich gezielt Gamification-Elemente auszuwählen.

\subsection{Leitmotive}

Als wesentlicher Teil der funktionalen Anforderungen aus \Cref{sec:funktionale-anforderungen} soll der Nutzer Spaß und Interesse an der Durchführung von Refaktorisierungen am eigenen Code haben. Eine Frustration des Nutzers durch wiederholtes Scheitern soll möglichst vermieden werden und kleine Fortschritte belohnt werden, um Frust zu vermeiden. Rätsel können die Neugier des Nutzers wecken und das Interesse an der Refaktorisierung amplifizieren. Der Octalysis-Ansatz bietet hierbei eine gute Richtlinie für die Auswahl von Gamification-Elementen und die Einschätzung von deren Wirkungsweisen auf den Nutzer.

\subsection{Narrativ}\label{sec:narrativ-konzept}

Beispielsweise können witzige und interessante Narrative dabei helfen, dass Nutzer sich noch mehr in die Thematik der Refaktorisierung vertiefen und dabei durch zwischenzeitliches Schmunzeln erfrischt werden. Ein solches Narrativ soll im Folgenden konzipiert werden. Wie in \Cref{sec:gamification-elemente} beschrieben, eignen sich sinnvolle Narrative für den Einsatz als Gamification-Element, um die im Game-Design zusammengestellten Aktionen miteinander zu verknüpfen und ihnen dabei Tiefe und Relevanz zu verleihen. Im ausgewählten Narrativ der Gamification-Erweiterung sollen die auf INLOOP eingereichten Lösungen als \enquote{Patienten} behandelt werden, wobei die Patienten krank oder gesund sein können und der Gesundheitszustand eines Patienten der Codequalität der dazugehörigen Lösung entspricht. Lösungen mit vielen detektierten schwerwiegenden Code Smells sind somit \enquote{todkranke} Patienten, während Lösungen mit hoher Konformität zu den festen Codierungsrichtlinien \enquote{gesund} sind. Der Nutzer nimmt hierbei die Rolle eines Arztes ein und versucht, seine eigenen Patienten (seine eingereichten Lösungen) von den Verletzungen der Codierungsrichtlinien zu heilen. Das Narrativ soll dabei aber keinesfalls leidvoll, sondern vor allem humoristisch orientiert sein, gestützt durch witzige Parallelen zur Realität.

\begin{figure}[H]
\centering
\includegraphics[width=0.75\linewidth]{narrative.png}
\caption{
    Ein Banner auf der Startseite, welches die \enquote{Code Doctors} vorstellt. Ursprüngliche Gestaltung der Illustration von Freepik\footnotemark.
}\label{fig:narrative}
\end{figure}
\footnotetext{Freepik. \url{https://www.freepik.com/free-vector/flat-nurse-team_4387619.htm} (Abgerufen am 14.9.2020)}

\noindent Eine dieser Parallelen soll durch die \enquote{Code Doctors}, also \enquote{Code-Ärzte} (nicht zu verwechseln mit Code-Dokumentation) geschaffen werden. Jeder Code Doc hat seine eigene, individuelle Profession. So gibt es einen allgemeinen Chirurgen, der sich zum Beispiel um gebrochene Knochen eines Patienten, also um die groben strukturellen Code Smells kümmert. Außerdem gibt es einen Schönheitschirurgen, welcher sich um die Schönheitsprobleme des Patienten, repräsentiert durch stilistische Code Smells, kümmert und einen Allgemeinmediziner, der verschiedene andere Probleme behandelt. Krankenschwestern helfen bei allem anderen, zum Beispiel beim Zurechtfinden in der Praxis, also die Erklärung von den einzelnen Gamification-Elementen, oder bei der Terminfindung, genauer der Initiierung einer \enquote{Untersuchung} (Code-Smell-Analyse) durch den Arzt. Jeder Code Doc hat dabei seinen eigenen Namen und einen kurzen Steckbrief, um dem gesamten Narrativ Glaubwürdigkeit zu verleihen.

Nutzer haben nun auf Grundlage dessen die Möglichkeit, ihre Lösungen einer ärztlichen Untersuchung zu unterziehen. Hierbei sollen sie sich in ein Wartezimmer begeben können, um die anwesenden Patienten zu sehen, wobei dies die jeweils aktuellsten Lösungen zu einer jeden Aufgabe sind. Gab es bereits eine \enquote{Voruntersuchung} zu einer vorherigen Lösung einer Aufgabe, so soll hierbei der Gesundheitszustand der vorherigen Variante angezeigt werden. Der Nutzer soll nun die Aufgabe übernehmen, den Lösungen zur \enquote{Genesung} zu verhelfen, indem er sich den gesundheitlichen Problemen (Code Smells) annimmt.

\subsection{Progressivität}\label{sec:progressivitaet}

Während der Nutzung von INLOOP soll Studierenden das Gefühl des Fortschrittes und der Verbesserung der eigenen Kompetenzen als psychologisches Grundbedürfnis (beschrieben in \Cref{sec:gamification-motivator}) vermittelt werden, um die intrinsische Motivation bei der Verbesserung der Codequalität zu amplifizieren. Hierzu sollen im Folgenden mögliche Gamification-Elemente vorgestellt werden, welche auf unterschiedliche Weise eine Progressivität vermitteln und verschiedene Faktoren aus dem nutzerorientierten Octalysis-Framework einbeziehen.

\subsubsection{Konsultationen}\label{sec:konsultationen}

Zur Einsicht der Code Smells, die den Gesundheitszustand einer Lösung beeinträchtigen, kann der Nutzer in der Rolle eines beobachtenden (angehenden oder lernenden) Doktors im Wartezimmer einen Patienten in eine Konsultation \enquote{hereinbitten}. In dieser Konsultation erhält der Nutzer eine Übersicht über den Code der Lösung und an welcher Stelle Code Smells detektiert wurden. Je nach der Art des jeweiligen Code Smells wird dieser nun dem Nutzer vom Code Doc übermittelt, der sich auf diesen (entsprechend zur Profession) spezialisiert hat. Dabei zeigt der Code Doc dem Nutzer, an welcher Stelle ein Code Smell detektiert wurde und wie er diesen beheben kann, narrativ repräsentiert durch ein \enquote{Rezept}.

\begin{figure}[H]
\centering
\includegraphics[width=\linewidth]{recipe.png}
\caption{Nutzer erhalten Rezepte von spezialisierten \enquote{Code Doctors} zur Verbesserung einer eingereichten Lösung.}\label{fig:recipe}
\end{figure}

\noindent \Cref{fig:recipe} zeigt, wie ein solches Rezept dem Nutzer präsentiert werden kann. Das Rezept nutzt die Nachricht aus der adaptierten Detektion des Codeanalysetools sowie die genaue Zeile und die Wichtigkeit der Detektion. Außerdem beinhaltet die Präsentation die aus der Erklärungsdatenbank bezogene Erklärung zu dieser Detektion. Da es sich hierbei um einen stilistischen Code Smell handelt, welcher als solcher in der Erklärungsdatenbank klassifiziert wird, ist der zuständige Code Doctor im Rahmen des Narratives der plastische Chirurg, da sich ein solcher vorrangig um das \enquote{Aussehen} des Patienten kümmert. Grobe strukturelle Verstöße werden hingegen vom Chirurgen behandelt, alle weiteren generellen Verstöße vom Allgemeinmediziner.

\paragraph{Code-Smell-Suche nach \cite{dos_santos_cleangame_2019}.} Durch das Aufzeigen des konkreten Fehlers erhält der Studierende die Möglichkeit, diesen direkt im Code zu beheben und damit seinen Patienten zu heilen. Die Konsultation kann jedoch bei Bedarf noch durch unterschiedliche Darstellungsstufen der Code Smells erweitert werden. Beispielsweise kann die Sichtbarkeit der Zeilennummer oder der konkreten Erklärung eingeschränkt werden. Hierdurch soll der Studierende, ähnlich zu dem in \Cref{sec:cleangame} vorgestellten Konzept, selbst herausfinden, an welcher Stelle eine Lösung Code Smells beinhaltet oder welche Art von Code Smell sich hinter einer konkreten Zeile versteckt. Zusammen hiermit kann nach Octalysis der Gamification-Faktor \enquote{Unpredictability and Curiosity} (siehe \Cref{fig:octalysis}) einbezogen werden. Um die Nutzererfahrung zu verstärken, kann so der Zufall auch visuell durch ein \enquote{Glücksrad} symbolisiert werden, wobei von mehreren möglichen Darstellungsvarianten des Code Smells eine Konkrete zufällig ausgewählt wird. Für eine genauere Auflistung könnte der jeweilige Code Doctor nochmals aufgefordert werden, genauere Untersuchungen auszuführen, sodass dann die jeweiligen Zeilennummern oder der Typ des Code Smells gezeigt werden, beispielsweise durch den Einsatz von Punkten.

\subsubsection{Punkte}

Um die Code Docs zu bitten, eine genauere Aufschlüsselung der Zeilennummern (durch weitere Untersuchungen) oder den konkreten Typ des Code Smells (abhängig vom Zufall) zu zeigen, muss der Nutzer hierfür Punkte ausgeben. Dies bedient die Gamification-Faktoren \enquote{Development and Accomplishment} sowie \enquote{Ownership and Possession} nach Octalysis. Die Punkte erhält ein Nutzer hierbei durch folgende Interaktionen:

\begin{itemize}
\item Durch das Einreichen von Lösungen, welche die Unit Tests der jeweiligen Aufgabe bestehen, abzüglich der Punkte, die als \enquote{Gesundheitskosten} aufgrund der vorliegenden Code Smells je nach Wichtigkeit hiervon abgezogen werden
\item Durch \enquote{Rückerstattung von der Krankenkasse}, wenn ein Rezept eingelöst wurde und der Code Smell eines Patienten beseitigt wurde
\item Durch das Freischalten von Badges, die in \Cref{sec:badges} weiter beschrieben sind
\end{itemize}

\noindent Die Möglichkeiten, Punkte zu erreichen, müssen für den Nutzer hierbei klar ersichtlich sein, um Verwirrung zu vermeiden. Erreicht ein Nutzer Punkte, so kann dies über entsprechende Notifikationen in INLOOP realisiert werden.

\subsubsection{Badges}\label{sec:badges}

Badges sind eine Möglichkeit, Punkte zu erhalten und werden analog zum Game-Design bei bestimmten Handlungen vergeben, die den Zielen der Gamification dienen und den zugrundeliegenden Spielregeln (Gamefulness) folgen. Im Folgenden sind mögliche Handlungen aufgelistet, die beispielsweise durch Badges weiter motiviert werden könnten:

\begin{itemize}
\item Das Betrachten bestimmter Informationsansichten, wie zum Beispiel einer Informationsansicht zu den Code Doctors anhand den von diesen behandelten Code Smells
\item Das Einreichen einer besonders guten Lösung, die keine oder nur wenige Code Smells beinhaltet
\item Das erstmalige Ausprobieren einer Funktionalität des Game-Designs, wie zum Beispiel das erstmalige Starten einer Konsultation
\item Die Nutzung von Möglichkeiten zur sozialen Interaktion mit anderen Studierenden
\end{itemize}

\noindent Im Zentrum steht hierbei vor allem die Belohnung der Auseinandersetzung mit besonderen Aktivitäten, die vom regulären Einreichen von Lösungen abweichen und auf die Kommunikation der Codierungsrichtlinien oder die generelle gegenseitige Motivation an der Bearbeitung von Aufgaben hinwirken. Gleichzeitig können die Aufgabenstellungen auch so vage formuliert sein, dass der Nutzer zum Rätseln animiert wird und herausfinden möchte, wie diese entsprechenden Badges erhalten werden können.

\subsubsection{Level und Fortschrittsbalken}

Ein weiteres progressives Gamification-Element kann durch Level repräsentiert werden. Je mehr Punkte ein Nutzer hat, desto höher ist sein Level. Fortschrittsbalken können dieses Level repräsentieren und auf das nächsthöhere hinweisen, um ein Bedürfnis zu erzeugen, dieses durch weitere Verbesserungen der Codequalität zu erreichen. Dabei können die erreichbaren Level gut in das Narrativ integriert werden. Entsprechend der Rahmenhandlung kann der Nutzer als Studierender anfangen und sich über die Ernennung als Doktor über Karrierestufen vom Facharzt über den Oberarzt bis hin zum Chefarzt oder ärztlichen Direktor verbessern.

\subsubsection{Progressive Leaderboards}\label{sec:leaderboards}

Level, Punkte und Badges sollen die Errungenschaften und den Fortschritt des Studierenden nach außen zeigen. In INLOOP ist es jedoch bisher nicht möglich, Informationen über andere Nutzer einzusehen.

\begin{figure}[H]
\centering
\includegraphics[width=\linewidth]{leaderboard.png}
\caption{UI-Konzept eines progressiven Leaderboards in Form einer Tabelle, die Nutzer nach ihren erreichten Punktzahlen auflistet.}\label{fig:leaderboard}
\end{figure}

\noindent Über progressive Leaderboards wird der Fortschritt des Studierenden für jeden anderen Nutzer der Plattform sichtbar. In Ergänzung zum bereits in anderen Gamification-Konzepten umgesetzten, rangbasierten Leaderboard wird dieses um eine progressive Komponente ergänzt, die in \Cref{fig:leaderboard} zu sehen ist und aus Rennspielen entnommen ist. Genauer ist hiermit die Rangänderung gemeint, die neben dem Rang im Leaderboard angezeigt wird. Sie ermöglicht nicht nur eine Momentanaufnahme des eigenen Rangs, sondern motiviert den Fortschritt nochmals, indem die Rangänderung (beispielsweise pro Tag) zusätzlich angezeigt wird.

\subsection{Soziale Interaktion}\label{sec:soziale-interaktion}

Bei der Betrachtung des Nutzungskonzeptes von INLOOP fällt auf, dass jeder Nutzer isoliert von anderen Nutzern Aufgaben bearbeitet und dadurch keine direkte soziale Interaktion auf der Plattform stattfindet. Die in der vorigen Sektion vorgestellten progressiven Leaderboards brechen diese Isolation auf, indem sie gezielt Informationen anderen Nutzern zugänglich machen und erstmalig eine soziale Interaktion auf INLOOP ermöglichen. Diese soziale Interaktion soll durch weitere Gamification-Elemente gestärkt werden und zu einem der zentralen intrinsischen Motivatoren nach dem Octalysis-Framework werden. Hierfür werden im Folgenden einige weitere Gamification-Elemente in das Konzept eingebracht.

\subsubsection{Avatare}

In \Cref{sec:gamification-elemente} wurde das Gamification-Element Avatare bereits vorgestellt. Nutzer sollen eine Auswahl an Avataren erhalten, wobei die Komplexität der Nutzerschnittstelle und der Auswahlmöglichkeiten variieren kann. Avatare sollen den Nutzer repräsentieren und es diesem ermöglichen, zum virtuellen Profil und dessen Fortschritt eine persönliche Beziehung aufzubauen. Eine weitere Möglichkeit wäre die Konfigurierbarkeit des Profils mit einem echten Profilbild, wobei allerdings sichergestellt werden muss, dass keine urheberrechtlich geschützten oder anstößigen Bilder hochgeladen werden können. Außerdem wahren Avatare die Privatsphäre des Nutzers, aus welchem Grund diese im Rahmen der Umsetzung zu präferieren sind.

\subsubsection{Auswahl von Kollegen}

Auf Grundlage von Avataren können weitere soziale Elemente integriert werden, beispielsweise eine Freundeslisten-Funktion. Im professionellen Kontext als Arzt im Rahmen des Narratives werden diese Freunde als \enquote{Kollegen} bezeichnet. Ein Nutzer hat hierbei die Möglichkeit, nach anderen Nutzern zu suchen und diese in seinem eigenen Profilbereich, in dem auch sein Avatar, Level, Badges und Punkte sichtbar sein können, als Kollegen aufzuführen und deren Fortschritt nachzuverfolgen. Hierbei kann dies so umgesetzt werden, dass der gewählte Kollege dies zunächst bestätigen muss, oder, so dass das Hinzufügen direkt geschieht (vgl. \Cref{sec:models}). Dies soll auch die Kompetitivität fördern, indem Nutzer dazu motiviert werden, ihre Kollegen, welche sie möglicherweise persönlich kennen, in ihrem Fortschritt oder Rang zu übertreffen.

\subsubsection{Fragen zu Lösungen anderer Nutzer}\label{sec:quizzes}

Weiteres Erkenntnispotenzial verbirgt sich hinter der Isolation des eingereichten Codes selbst. Bedingt dadurch, dass Nutzer Lösungen anderer nicht plagiieren sollen, erhalten Nutzer lediglich Zugang zum Quellcode der \textit{eigenen} Lösungen. Analog zu den Code-Smell-Quizzes aus \cite{dos_santos_cleangame_2019} könnte es von Vorteil sein, wenn Nutzer Code anderer lesen und verstehen müssen, um so zu verstehen, wie Code Smells die Lesbarkeit von Lösungen verringern können. Um zu vermeiden, dass Nutzer Code aus fremden Lösungen verwenden, um selbst Aufgaben zu bestehen, könnte ein Nutzer beispielsweise nur Zugang zu Lösungen bekommen, die er selbst schon bestanden hat. Analog zu den in \Cref{sec:cleangame} beschriebenen CleanGames könnte ein Code-Smell-Quiz zu diesen Lösungen integriert werden. Im Rahmen des Narratives könnten hierbei Patienten untersucht werden, die aufgrund ihres schlechten Gesundheitszustands gestorben sind (vgl. \Cref{sec:scarcity}).

\subsection{Risiko und zeitliche Knappheit}\label{sec:scarcity}

Wie in der vorigen Sektion erwähnt, sollen Patienten auch \enquote{sterben} können, wenn diese zu viele Code Smells beinhalten oder über eine längere Zeit nicht geheilt werden konnten. Stirbt ein Patient zu einer Aufgabe, so hat ein Nutzer nicht mehr die Möglichkeit, die Punktzahl zu dieser Aufgabe zu verbessern. Als Bedingung, ob eine Lösung als \enquote{gestorben} gekennzeichnet wird, kann die Punktzahl herangezogen werden. Erreicht eine Lösung keine Punkte, also werden von der erreichbaren Punktzahl so viele Punkte für die Code Smells abgezogen, dass keine Punkte übrig bleiben, so könnte sie als gestorben gekennzeichnet werden. Dieses Konzept repräsentiert die Gamification-Komponente \enquote{Loss and Avoidance} aus dem Octalysis Framework, indem ein Nutzer nun nicht mehr unbegrenzt Zeit hat, einen Patienten zu heilen, oder der Patient auch sofort sterben kann, wenn der Nutzer nicht auf eine hinreichende Konformität achtet. Entsprechend müssen die erreichbaren Punktzahlen der Aufgaben so gewählt werden, dass dieser Sonderfall nicht zu häufig auftritt, um mögliche Frustrationen bei den Nutzern zu vermeiden. Beispielsweise könnte die Punktzahl für Exam-Aufgaben in INLOOP zunächst auf 3000 bis 4000 Punkte festgesetzt und anschließend anhand der statistischen Daten nachjustiert werden. Außerdem könnte der Punktabzug für wiederholte Code Smells desselben Typs, zum Beispiel das wiederholte Vergessen von Klammern, graduell verringert werden, um zu vermeiden, dass das häufige Auftreten eines Fehlers zu übermäßigem Punktabzug führt.

\subsection{Dialogkonzept}

Um die konzeptuellen Elemente der vorigen Sektionen anzuwenden und hierfür in konkrete bestehende Dialoge von INLOOP zu integrieren, wurde eine Dialoglandkarte angefertigt.

\begin{figure}[H]
\centering
\includegraphics[width=\linewidth, bb=0 0 555 232]{aktivitaeten.pdf}
\caption{Konkrete beispielhafte Dialoglandkarte für die Integration der Gamification-Elemente in bestehenden Dialogen von INLOOP.}\label{fig:aktivitaeten}
\end{figure}

\noindent Die in \Cref{fig:aktivitaeten} gezeigte Dialoglandkarte zeigt den bestehenden zentralen INLOOP-Workflow, bei dem ein Nutzer sich zunächst einloggt, dann eine Aufgabe über die Aufgabenliste wählt, diese in einem Editor mit der Aufgabenbeschreibung bearbeitet und einreicht, wonach er seine eingereichten Lösungen sehen und diese gegebenenfalls reiterieren kann, beispielsweise, wenn die Unit-Tests oder der Kompilierprozess nicht erfolgreich waren. Hinzu kommen nun weitere, optionale Dialoge der Gamification-Erweiterung, beginnend mit einer Darstellung der erreichbaren Punkte in der Aufgabenliste und der Möglichkeit, in einer Wiki-Ansicht bestimmte Informationen über die Gamification-Erweiterung zu erhalten, zum Beispiel die Möglichen Verstöße inklusive deren Erklärungen. Außerdem können von hieraus auch die in \Cref{sec:quizzes} erläuterten Quizzes gestartet werden. Führt ein Nutzer hierbei bereits eine Aktion aus, die dazu führt, dass er eine Errungenschaft erhält, wird über einen modularen Notifikationsdialog die dazugehörige Notifikation eingeblendet. Dies ist jedoch unabhängig von der aktuellen Ansicht, so dass Notifikationen prinzipiell in jeder Ansicht eingeblendet werden können. Gibt ein Nutzer eine Aufgabe ab, so hat er zusätzlich die Möglichkeit, die erreichten Punkte der Abgabe zu sehen. Nun hat er die Möglichkeit, die Lösung direkt zu überarbeiten, beispielsweise, wenn der Kompilierungsprozess gescheitert ist. Zusätzlich jedoch, wenn alle Tests erfolgreich waren, kann der Nutzer seine Lösung über einen gesonderten Warteraum oder direkt in einer Konsultation bezüglich der detektierten Code Smells analysieren und von dort aus direkt zur Überarbeitung der Lösung gelangen, um mehr Punkte zu erhalten. Seinen Gesamtfortschritt kann der Nutzer in einer Fortschrittsansicht einsehen. Hierhin gelangt der Nutzer direkt über die Navigationsleiste oder über die progressiven Leaderboards, in denen sich der Nutzer mit anderen vergleichen kann. In der Fortschrittsansicht selbst kann der Nutzer seine (in den Leaderboards) gewählten Kollegen anzeigen lassen und selbst einen Avatar auswählen beziehungsweise auch seine erreichten und erreichbaren Errungenschaften anzeigen lassen. Die unterschiedlichen Dialoge haben unterschiedliche Intentionen, so dass zum Beispiel die Wiki-Ansicht zur Information über Codequalität beiträgt, die Komponenten der Progressivität zusätzliche Kompetitivität über das Punktesystem induzieren und schließlich Konsultationen über eigene Verstöße aufklären sollen und gleichzeitig über eine Reiteration der Lösung zum Gewinn von mehr Punkten (adaptiert aus \cite{haendler_serious_2019}) und dadurch zur Motivation der Verbesserung der Codequalität beitragen sollen.

\subsection{Einordnung nach Octalysis}

Die beschriebenen Gamification-Elemente des Game-Designs wurden so ausgewählt, dass die Gamification-Faktoren nach Octalysis hiervon vollständig abgedeckt werden, um eine möglichst ausgewogene Gamification zu erreichen.

\begin{figure}[H]
\centering
\includegraphics[width=0.75\linewidth, bb=0 0 400 397]{octalysis-concept.pdf}
\caption{Einordnung der gewählten Gamification-Elemente nach dem Octalysis Gamification Framework von Chou \cite{chou_actionable_2019}.}\label{fig:octalysis-concept}
\end{figure}

\noindent In \Cref{fig:octalysis-concept} wird dies sichtbar. Das konzipierte Game-Design verwendet ausgewählte Gamification-Elemente, die sowohl extrinsisch (Avatare, Kauf mit Punkten) als auch intrinsisch (Soziale Interaktion) wirken. Außerdem sind sowohl negativ wirkende Elemente (Risiko und zeitliche Knappheit) als auch positiv wirkende Elemente, wie das allgegenwärtige Narrativ, enthalten. Das Narrativ dient als verbindendes Element zwischen den einzelnen Gamification-Elementen und leitet den Nutzer durch die Anwendung, mit Hinblick auf die Motivation der Refaktorisierung über die Erkennung und Behebung von Code Smells.

\subsection{Einordnung nach Bloom}

Wie in \Cref{tab:bloom} und der dazugehörigen Sektion diskutiert, liegt ein zentraler Aspekt in der Kreation einer didaktisch wirkungsvollen Anwendung in der Evaluation der notwendigen Kompetenzen. Anhand der von Bloom et al. hierfür entwickelten Taxonomie soll auch für das vorliegende Game-Design berücksichtigt werden, dass dieses auch die höheren Lernziele (wie der Ansatz von Händler und Neumann) erfüllen kann. Die Möglichkeit, Refaktorisierungskandidaten zunächst selbst zu suchen und dann gegebenenfalls Hilfe (gegen den Handel von Punkten) zu erhalten, Code anderer Nutzer zu lesen und zu verstehen sowie selbst Clean Code zu entwickeln und dafür Punkte zu erhalten, zielt ab auf unterschiedliche Lernziele nach Blooms Taxonomie, so dass der Nutzer selbst zunächst die Fakten und Zusammenhänge zwischen Clean Code und Code Smells verstehen müssen, diese anhand seiner eigenen Lösungen und der Lösung anderer analysieren und schließlich diese Prinzipien auf die weitere Programmierung selbstständig anwenden soll.


\section{Architekturentwurf der Gamification-Erweiterung}

Auf Grundlage der Architektur zur Integration der statischen Codeanalyse sowie der Ergänzung von Code Smells um semantische Beschreibungen und dem vorgestellten Game-Design soll nun die Gamification-Erweiterung architekturell entworfen werden.

\subsection{Integrationspunkte und Komponenten}

Da die Gamification-Erweiterung in das bestehende INLOOP-System integriert werden soll, muss zunächst eine Konzeption der konkreten Integrationspunkte durchgeführt werden, um besser zu verstehen, an welchen Stellen die Erweiterung ansetzt und die Erweiterung bestmöglich von der Basisanwendung hinsichtlich der Modularität und der Wartbarkeit abzukapseln.

\subsubsection{Komponentenansicht}

\begin{figure}[H]
\centering
\includegraphics[width=\linewidth, bb=0 0 684 390]{components.pdf}
\caption{Übersicht über mögliche Komponenten der Gamification-Erweiterung und deren Integrationspunkte in der bestehenden INLOOP-Architektur.}\label{fig:components}
\end{figure}

\noindent Anhand von \Cref{fig:components} werden die existierenden Komponenten in INLOOP sichtbar, an welche die Teilkomponenten der Gamification-Erweiterung, welche unter dem Metamodul \textbf{inloop.medics} zusammengefasst sind, anknüpfen:

\begin{itemize}
\item \textbf{inloop.solutions} ist für die Prozesse der Lösungseinreichung und -darstellung verantwortlich und stellt Lösungsdateien bereit.
\item \textbf{inloop.testrunner} beinhaltet die TestRunner Architektur und erhält von inloop.solutions Lösungsdateien, welche durch dieses Modul getestet werden.
\item \textbf{inloop.gitload} stellt die Continuous-Publishing-Funktionalitäten bereit und die darüber bezogenen Docker Images, von denen die vorkonfigurierten Docker Container zur Durchführung der Tests im TestRunner instanziiert werden.
\item \textbf{inloop.tasks} ist als Modul verantwortlich für die in INLOOP angebotenen Aufgabenstellungen.
\item \textbf{inloop.statistics} ermöglicht die Integration von Statistiken für den Administrator.
\item \textbf{inloop.accounts} stellt die grundlegenden Registrierungs- und Login-Funktionalitäten bereit und erweitert diese um weitere Funktionalitäten, wie beispielsweise das Speichern einer Matrikelnummer.
\item \textbf{inloop.grading} wird genutzt für die Vergabe von Bonuspunkten, welche für diese Erweiterung nicht vorgesehen ist. Daher ist dieses Modul in \Cref{fig:components} nicht aufgeführt.
\end{itemize}

\noindent Im Folgenden sollen die neuen Komponenten anhand deren Aufgaben beschrieben werden.

\paragraph{inloop.medics.violations:} Diese Komponente beinhaltet die vorgestellte Parser-Architektur aus \Cref{fig:integration2} und realisiert hierüber das Einlesen des Test-Outputs der Komponente inloop.testrunner und die Abbildung der Detektionen auf Violations. Die Komponente stellt diese Violations zur Verfügung, indem die dazugehörigen Lösungsdateien (durch inloop.solutions bereitgestellt) im Rahmen einer Konsultation bereitgestellt werden. Des Weiteren ist diese Komponente verantwortlich für die Bewertung der Lösungen. Dazu bezieht sie von der inloop.tasks Komponente die jeweils erreichbaren Punkte für eine Aufgabe, wobei dies im Rahmen der Integration hinzuzufügen ist. Die erreichbaren Punkte können zum Beispiel vom Schwierigkeitsgrad der Aufgabe abhängen. Die Aufgabenbewertung wird als Schnittstelle der Komponente nach außen verfügbar gemacht. Außerdem stellt die Komponente erkannte Violations für die Analyse der häufig auftretenden Code Smells in der inloop.statistics Komponente zur Verfügung, analog zur Anwendungsfall- und Anforderungsspezifikation aus \Cref{sec:use-cases-actors} und \Cref{sec:spezifikation}.

\paragraph{inloop.medics.rewards:} Mithilfe dieser Komponente erhalten Nutzer die Möglichkeit, Errungenschaften zu erhalten, zum Beispiel in Form von Badges. Die Isolation dieser Komponente ist besonders schwierig, weil die Ereignisse, welche zur Freischaltung einer Errungenschaft führen können, in verschiedenen anderen Komponenten erzeugt werden können. In \Cref{fig:components} ist die Einreichung einer Lösung in inloop.solutions und das Erreichen einer Punktzahl in inloop.medics.violations als Beispiel hierfür aufgeführt. Die Komponente muss entsprechende Schnittstellen bereitstellen, so beispielsweise die Prüfung der Freischaltung einer Badge möglichst selbstständig durchführen können. Hierfür kann das Observer\cite[S. 448ff]{ullenboom_java_2020} Entwurfsmuster oder eine Abwandlung genutzt werden. Django stellt als Implementation dessen das event-basierte Signals\footnote{Django Signals. \url{https://docs.djangoproject.com/en/3.1/topics/signals/} (Abgerufen am 16.9.2020)} Framework zur Verfügung. Wird eine Badge durch diesen Prozess freigeschaltet, so stellt die Komponente eine entsprechende Notifikation zur Verfügung sowie die hierbei gutgeschriebene Punktzahl.

\paragraph{inloop.medics.ranking:} Die Punktzahlen von Lösungen und von Badges werden Nutzern über diese Komponente gutgeschrieben und in einem Leaderboard präsentiert, wobei die Komponente den individuellen Rang präsentiert. Dabei werden die Punkte für jeden Nutzer aggregiert und daraus eine Rangfolge erstellt. Da diese Aufgabe rechenintensiv ist, sollte sie gesondert ausgeführt werden (siehe \Cref{sec:auslagerung}). Die Aggregation kann hierbei in diskreten Zeitintervallen geschehen. Um zusätzlich die Rangänderungen zu aggregieren, bietet sich beispielsweise ein Zeitintervall von 24 Stunden an, sodass sich Ränge und Rangänderungen täglich aktualisieren.

\paragraph{inloop.medics.social:} Diese Komponente ermöglicht die Bereitstellung der sozialen Funktionalitäten und der Assoziation des Rangs aus der inloop.medics.ranking Komponente mit einem konkreten Spieler. Hierzu wird der jeweils zu einem Spieler gehörende Nutzer von der inloop.accounts Komponente bezogen und mit weiteren Merkmalen ausgestattet. Beispielsweise erhält ein Spieler die Möglichkeit, einen konkreten Avatar auszuwählen. Außerdem stellt diese Komponente die Funktionalität bereit, dass Spieler andere Spieler verfolgen können, indem jeder Spieler im Verantwortungsbereich dieser Komponente andere Spieler zu seinen Kollegen ernennen kann.
\\

\noindent Gemeinsam bilden die oben genannten Komponenten die Architektur der Gamification-Erweiterung. Durch die Komponentenansicht konnten allerdings noch nicht alle Integrationspunkte hinreichend erläutert werden. Hierzu werden in den folgenden Sektionen nochmals konkrete Teile der Architektur aufgegriffen, die eine besondere Rolle im Integrationskonzept spielen.

\subsubsection{Anpassung der Build-Konfiguration}

Um die Integration der Codeanalysetools in den containerisierten Buildprozess zu ermöglichen, ist es notwendig, das über Continuous Publishing bezogene Docker Image mit den entsprechenden Build-Schritten für die auszuführenden Codeanalysetools zu ergänzen.

\begin{figure}[H]
\centering
\includegraphics[width=\linewidth, bb=0 0 627 210]{build-config.pdf}
\caption{Notwendige Anpassungen (grau hinterlegt) im Ant Build-Prozess des über Continuous Publishing bezogenen Docker Image des TestRunners.}\label{fig:build-config}
\end{figure}

\noindent Diese Schritte sind in \Cref{fig:build-config} als grau hinterlegt markiert. Hinzu kommen die hier nicht explizit gezeigten Konfigurationen aus dem QualityReview Framework. Die Änderungen müssen im durch Continuous Publishing bezogenen Aufgaben-Repository realisiert werden. Hierfür ist es notwendig, die Archive der Codeanalysetools im Docker Image (grau hinterlegt) zu integrieren, damit diese im vom TestRunner instanziierten Docker Container ausgeführt werden können.

\subsubsection{Auslagerung rechenintensiver und periodisch wiederkehrender Aufgaben}\label{sec:auslagerung}

INLOOP nutzt Service Workers, also von der eigentlichen Webanwendung separierte Prozesse, um rechenintensive und periodisch wiederkehrende Aufgaben wie das Durchführen eines Tests durch den TestRunner vom klassischen Request-Response Zyklus oder das Löschen von ungültigen Nutzern auszugliedern. Die Gamification-Erweiterung inkludiert selbst rechenintensive und periodisch wiederkehrende Prozesse, welche aus dem Request-Response Zyklus ausgegliedert werden müssen. Für die Berechnung der Leaderboards beispielsweise sind komplexe Aggregationen für jeden Nutzer notwendig, um den insgesamten Punktestand zu ermitteln. Außerdem ist dieser Prozess unabhängig von einer konkreten Anfrage an die Webanwendung. Für die Integration bietet sich daher an, diesen Prozess an einen Service Worker Prozess auszulagern. Idealerweise werden diese Berechnungen auch nur zu einem Zeitpunkt ausgeführt, wenn die Webanwendung nur eine geringe Last besitzt, zum Beispiel um Mitternacht.

\subsubsection{Bereitstellung von Notifikationen}\label{sec:notifikationen}

Zur Ermöglichung eines direkten Feedbacks, sollen Nutzer Notifikationen zu bestimmten ausgeführten Aktionen erhalten, beispielsweise die Änderung ihres Punktestandes. Das im Django Framework, auf welchem die webanwendungsspezifischen Funktionalitäten INLOOPs basieren, bereitgestellte und bereits in INLOOP genutzte Messages-Framework ist hierzu nicht applizierbar, weil dies nur im nicht unterbrochenen Request-Response-Zyklus eingebunden werden kann. Bei an Service Worker ausgelagerten Prozessen oder bei von Django Signals initiierten Prozessen, welche jeweils eine Notifikation zur Folge hätten, könnte dies nicht genutzt werden. Daher ist die Einführung einer weiteren Komponente notwendig, welche zuvor persistierte Notifikationen in die Webanwendung integriert. Für diese Komponente sollen nun zwei konkrete Konzepte vorgestellt werden.

\begin{figure}[H]
\centering
\includegraphics[width=0.75\linewidth, bb=0 0 476 181]{notification-middleware.pdf}
\caption{Architektur für die Bereitstellung von Notifikationen im Pull-Prinzip über eine Django-Middleware oder einen Django-Kontextprozessor im Request-Response-Prozess.}\label{fig:notification-middleware}
\end{figure}

\noindent Das erste Konzept ist in \Cref{fig:notification-middleware} gezeigt und beinhaltet die Einführung einer Komponente, die ähnlich wie ein Proxy agiert. Browser-Anfragen eingeloggter Nutzer an den Server werden über diesen Proxy geleitet, wobei die INLOOP-Datenbank nach ungelesenen Notifikationen für einen Nutzer abgefragt wird. Diese Notifikationen können hier zuvor von anderen Prozessen erzeugt worden sein und sind nicht an den Request-Response-Prozess gebunden. Geladene, ungelesene Notifikationen werden als gelesen markiert und in Metadaten der Anfrage des Nutzers geschrieben. Diese Metadaten werden von der angeforderten Ansicht eingelesen und können nun verwendet werden, um die Notifikationen in das zurückzugebende HTML zu rendern. Der Nutzer erhält die Notifikation in der angeforderten Seite. Zu exkludieren sind hierbei Anfragen, die kein HTML, sondern beispielsweise Ressourcen anfordern, beispielsweise AJAX Anfragen. Dieses Prinzip ist sehr einfach integrierbar über die Nutzung einer Django Middleware\footnote{Django Middleware. \url{https://docs.djangoproject.com/en/3.1/topics/http/middleware/} (Abgerufen am 16.9.2020)}, welche es ermöglicht, im Request-Response Zyklus zusätzliche Operationen zu ergänzen, oder über das Verwenden eines Django Kontextprozessors\footnote{Django Template API. \url{https://docs.djangoproject.com/en/3.1/ref/templates/api/} (Abgerufen am 16.9.2020)} für die Ergänzung von zusätzlichen Umgebungsinformationen, in diesem Fall die Notifikationen. Leider erhält der Nutzer auf diese Weise Notifikationen immer erst beim Laden einer Seite, nicht schon asynchron, während die Seite geöffnet ist. Daher soll noch ein zweites Konzept vorgestellt werden.

\begin{figure}[H]
\centering
\includegraphics[width=0.75\linewidth, bb=0 0 390 189]{notification-middleware-push.pdf}
\caption{Architektur für die Bereitstellung von Notifikationen im Push-Prinzip über einen asynchronen Nachrichtenaustausch.}\label{fig:notification-middleware-push}
\end{figure}

\noindent Bei dem in \Cref{fig:notification-middleware-push} vorgestellten zweiten Konzept zur Bereitstellung von Notifikationen wird zur Bereitstellung von Notifikationen ein Seitenkanal benutzt, der von dem Django-Channels-Framework bereitgestellt werden kann. Hierbei registriert sich der Nutzer bei dem Besuch der Seite über eine WebSocket Verbindung in einem von Django bereitgestellten individuellen \enquote{Consumer}. Sobald ein nebenläufiger Prozess eine Notifikation erzeugt, wird diese über den Nachrichten-Bus an den Consumer übermittelt, und schließlich über die aktive WebSocket-Verbindung an den Nutzer. Eine Persistierung ist nicht zwingend notwendig, jedoch sinnvoll, um Notifikationen nicht zu verlieren. Dieser Ansatz ermöglicht eine Benachrichtigung des Nutzers in Echtzeit, jedoch ist der Ansatz technisch limitiert und erfordert den Einsatz eines ASGI\footnote{ASGI. \url{https://asgi.readthedocs.io/en/latest/} (Abgerufen am 13.8.2020)}-unterstützenden Server-Backends, wie zum Beispiel Daphne\footnote{Daphne. \url{https://github.com/django/daphne} (Abgerufen am 13.8.2002)}, und damit eine Umstrukturierung des Deployments.

\subsection{Modelle und Persistenzschicht}\label{sec:models}

Zusätzlich zur Strukturierung der Gamification-Erweiterung in Teilkomponenten soll nun auch eine konkrete jeweilige Modellstruktur entworfen werden, als Grundlage für die Implementation im nächsten Kapitel.

\begin{figure}[H]
\centering
\includegraphics[width=\linewidth, bb=0 0 735 247]{ekd-violations.pdf}
\caption{Eine mögliche Modellstruktur der inloop.medics.violations Komponente.}\label{fig:ekd-violations}
\end{figure}

\noindent \Cref{fig:ekd-violations} zeigt eine Modellstruktur, auf deren Grundlage die Funktionalitäten der inloop\allowbreak .medics\allowbreak .violations Komponente umgesetzt werden können. Hierbei werden, wie in den vorigen Sektionen erläutert, Violations durch das Parsing der Reports von Codeanalysetools aus dem TestRunner erstellt und zu einem jeweiligen SolutionFile aus der inloop.solutions Komponente zugewiesen, um dieses später in der Nutzeroberfläche zusammen anzeigen zu können. Außerdem erhält eine Violation eine Priorität in Form einer zu validierenden Zeichenkette, wobei auf Grundlage dieser später der Punktabzug für Lösungen über die Bereitstellung einer Penalty errechnet werden muss sowie eine Rule, welche die Abbildung der Erklärungsdatenbank symbolisiert und die jeweiligen Erklärungen beinhaltet. Eine Rule wird analog zum Narrativ einem Medic zugeordnet, welcher einen Namen und eine spezielle Profession (zum Beispiel Chirurg) hat. Zu beachten ist, dass die Rule-Objekte und die Medic-Objekte nicht dynamisch vom Nutzer erzeugt werden, sondern zum Start der Anwendung über bestimmte Migrationsmechanismen bereitgestellt werden müssen. Außerdem ist anzumerken, dass, wie bei allen in dieser Sektion gezeigten bidirektionalen Assoziationen die Gegenseite durch objektrelationales Datenbankmapping automatisch inferiert wird und somit in der Integration implizit gegeben ist, aber in der Modellstruktur dennoch explizit notiert ist. Der Gesundheitszustand einer Lösung bestimmt sich durch die Anzahl an Violations und der durch die Aufgabe der Lösung erreichbare Punktzahl, von der diese abgezogen werden. Erreicht eine Lösung die volle mögliche Punktzahl, so ist sie gesund. Erhält sie Punkte, jedoch nicht alle möglichen Punkte, so ist die Lösung krank. Eine Lösung wird als gestorben gekennzeichnet, wenn sie keine Punkte erreicht hat. Auf Grundlage dessen können außerdem die im Game-Design erwähnten Konsultationen (\Cref{sec:konsultationen}) oder Fragen zu Lösungen anderer (\Cref{sec:quizzes}) umgesetzt werden, wozu eine Konsultation oder ein Quiz zu mehreren Violations erstellt und mit den zur jeweiligen Violation sichtbaren Informationen ausgestattet werden kann.

\begin{figure}[H]
\centering
\includegraphics[width=\linewidth, bb=0 0 576 259]{ekd-social.pdf}
\caption{Eine mögliche Modellstruktur der inloop.medics.social Komponente.}\label{fig:ekd-social}
\end{figure}

\noindent Die in \Cref{fig:ekd-social} gezeigte inloop.medics.social Komponente erweitert durch das PlayerDetails Modell den bereits gegebenen User, wobei dieser durch die erreichten Punkte, dem direkt damit zusammenhängenden Level und einem auswählbaren (jedoch optionalen) Avatar ergänzt wird. Außerdem ermöglicht die Komponente das Erstellen von so genannten ColleagueTrackers, welche die Kollegen-Beziehung eines Nutzers zu einem anderen abbildet. Hierbei gibt es immer einen \enquote{tracker}, also der Nutzer, welcher die Kollegenbeziehung mit dem hierbei passiven \enquote{tracked\_colleague} auswählt. Wählt ein Nutzer einen anderen Nutzer als Kollegen aus, so bedeutet dies nicht, dass diese Beziehung auch umgekehrt gelten muss. Somit wird verhindert, dass Nutzer gegen ihren Willen auf dem eigenen Profil als Kollege eines anderen, möglicherweise ihnen unbekannten Nutzer, hinzugefügt werden. Für eine implizite Umkehrbarkeit wäre die Konzeption und Integration von Freundschaftsanfragen sinnvoll.

\begin{figure}[H]
\centering
\includegraphics[width=\linewidth, bb=0 0 526 318]{ekd-rewards.pdf}
\caption{Eine mögliche Modellstruktur der inloop.medics.rewards Komponente.}\label{fig:ekd-rewards}
\end{figure}

\noindent Eine zentrale Rolle in der inloop.medics.rewards Komponente spielen die Badges, welche durch einen Spieler freigeschaltet werden können. Badges werden hierbei durch einen Identifikator gekennzeichnet, durch welchen diese später in der Anwendungslogik referenziert werden können. Über eine Schnittstelle, die für einen Nutzer eine Badge mit einem bestimmten Identifikator durch eine spezielle Interaktion freischaltet, ist eine hinreichende Entkopplung vom Kontext der anderen Komponente möglich. Hierbei ist zu unterscheiden zwischen Badges, welche unabhängig vom Nutzer über den initialen Migrationsprozess geladen werden können (repräsentiert durch das Badge Modell) sowie der Markierung, dass ein bestimmter Nutzer eine Badge freigeschaltet hat (BadgeScore). Bei der Freischaltung einer Badge wird über eine der in \Cref{sec:notifikationen} diskutierten Architekturen eine Notifikation erstellt und an den Nutzer ausgegeben.

\begin{figure}[H]
\centering
\includegraphics[width=\linewidth, bb=0 0 691 433]{ekd-ranking.pdf}
\caption{Eine mögliche Modellstruktur der inloop.medics.ranking Komponente.}\label{fig:ekd-ranking}
\end{figure}

\noindent Das BadgeScore-Modell stammt hierbei aus der inloop.medics.ranking Komponente, wie in \Cref{fig:ekd-ranking} gezeigt. Diese beinhaltet verschiedene Modelle, die als Datenpunkte für die Persistierung der erreichten Punkte eines Nutzers dienen. Das Metamodell, Score, referenziert hierfür die dazugehörigen PlayerDetails und speichert einen ganzzahligen Punktwert. Da sowohl die Abgabe von Lösungen, als auch das Freischalten von Badges dem Nutzer Punkte gutschreibt, kann ein Score sowohl ein SolutionScore (referenziert Lösung), als auch ein BadgeScore (referenziert Badge) sein. Diese Score-Objekte werden durch die Interaktionen des Nutzers erzeugt und mit den entsprechenden Punkewerten versehen. Da beispielsweise eine Lösung auch aktualisiert werden kann und die Punktwerte nur für die aktuellste, bestandene Lösung gutgeschrieben werden sollen, wurde dem Metamodell Score noch ein boolsches Feld hinzugefügt, um die Validität zu setzen. SolutionScore Objekte überholter Lösungen können auf invalid gesetzt werden und werden infolgedessen nicht mehr für die Punkteberechnung herangezogen. Des Weiteren können Payments eingeführt werden, für die Umsetzung der Kauf-Mechanik in Konsultationen, um eine Detektion aufzudecken. Diese Payments würden dann von der Punktzahl eines Nutzers abgezogen werden, realisierbar durch einen negativen Punktwert. Die in Fragerunden zu Lösungen anderer Nutzer (\Cref{sec:quizzes}) richtig beantworteten Fragen erzielen Punkte als QuizScore. Die Punkte eines Nutzers können somit errechnet werden, indem die Summe aller Punkte der validen Score Objekte eines Nutzers gebildet wird. Anhand dessen wird auch das Level des Nutzers bestimmt. Für die Realisation der dynamischen Leaderboards wird außerdem ein RankSnapshot Modell eingeführt, welches für einen Nutzer in diskreten Zeitintervallen den Rang berechnet und den Unterschied zur letzten Position abspeichert.
\\

\paragraph{Skalierbarkeit:}\label{p:skalierbarkeit} Die Erstellung der RankSnapshots erfordert in festen Zeitabständen wiederholte und rechnenintensive Aggregationsoperationen über der Datenbank. Dies ist bei der Implementation zu beachten und gesondert an Service-Worker-Prozesse auszulagern. Außerdem ist eine konsistente Zwischenspeicherung der Punkte eines Nutzers sinnvoll, wenn Leaderboards nach diesem Kriterium sortiert werden sollen.


\section{Zusammenfassung}

Die vorgestellte Architektur zur Integration des QualityReview Framework und Vereinheitlichung von Code Smells, zusammen mit dem Einsatz einer Erklärungsdatenbank, schafft eine konzeptionelle Grundlage für die Implementation der Gamification-Erweiterung. Hierzu werden die aufbereiteten Detektionen im Rahmen einer Konsultation oder eines Quiz an den Nutzer vermittelt, wobei dies Teil eines in eine Rahmenhandlung integrierten Game-Designs ist. Zur weiteren Motivation des Nutzers wurden konkrete Gamification-Elemente auf Grundlage des Octalysis-Frameworks ausgewählt und konzeptuell in das Narrativ integriert. Unter Berücksichtigung der psychologischen Wirkung werden hierdurch die acht Kernantriebe der Octalysis genutzt, um eine ausgewogene Gamification mit Hinblick auf die Motivation von Clean Code zu realisieren. Nutzer können zum Beispiel Punkte, Errungenschaften und Level durch das aktive Lernen der Charakteristika von Clean Code, zum Beispiel durch die Minimierung der Code Smells in eigenen Lösungen, erreichen und sich mit anderen Nutzern in einem sozialen Kontext vergleichen. Das direkte Feedback wird durch die Bereitstellung von Notifikationen erweitert. Nachfolgend wurde konzeptionell gezeigt, wie die Gamification-Elemente des Game-Designs in INLOOP integriert werden können und welche Teilkomponenten hierbei separierbar sind. Für die einzelnen Teilkomponenten wurden konkrete Modelle konzipiert, welche nun für die prototypische Implementation der Gamification-Erweiterung verwendet werden können.
