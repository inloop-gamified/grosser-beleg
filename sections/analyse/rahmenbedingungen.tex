\section{Rahmenbedingungen}\label{sec:rahmenbedingungen}

Da die Konzeption der Erweiterung von INLOOP abhängig ist, soll nachfolgend zunächst dessen Iststand beschrieben, eine Zielgruppenbeschreibung ermittelt sowie konkrete Anwendungsfälle gezeigt werden.

\subsection{Istzustand von INLOOP}

\begin{figure}[H]
\centering
\includegraphics[width=0.5\linewidth, bb=0 0 216 163]{inloop-architecture.pdf}
\caption{Architektur-Metamodell von INLOOP. Entnommen aus \cite{morgenstern_continuous_2018}.}\label{fig:inloop-architecture}
\end{figure}

\noindent INLOOP ist eine auf Python\footnote{Python. \url{https://www.python.org/} (Abgerufen am 15.9.2020)}, Git\footnote{Git. \url{https://git-scm.com/} (Abgerufen am 15.9.2020)}, dem Django Framework\footnote{Django. \url{https://www.djangoproject.com/} (Abgerufen am 15.9.2020)} und Docker\footnote{Docker. \url{https://www.docker.com/} (Abgerufen am 15.9.2020)} basierende Webanwendung. In \Cref{fig:inloop-architecture} ist die äußere Struktur der Anwendung gezeigt. Der Client kann direkt auf die Webanwendung zugreifen, wobei diese mit weiteren Komponenten zur Realisation der einzelnen Funktionalitäten kommuniziert. Zur persistenten Datenspeicherung wird eine Datenbank verwendet. Um asynchrone Prozesse zu planen und auszuführen, wird ein Job Broker verwendet, welcher Hintergrundarbeiter initiiert. Zu diesen asynchronen Aufgaben gehören beispielsweise die kontinuierliche Integration von Aufgaben über ein externes Aufgaben-Repository, aber auch die Steuerung des Aufgaben-Testprozesses in dedizierten containerisierten Laufzeitumgebungen.

\begin{figure}[H]
\centering
\includegraphics[width=\linewidth, bb=0 0 481 333]{inloop-components.pdf}
\caption{UML-Kompositionsstrukturdiagramm: Top-Level-Architektur der Webanwendung INLOOP.}\label{fig:inloop-components}
\end{figure}

\noindent Das Kompositionsstrukturdiagramm in \Cref{fig:inloop-components} zeigt die Komponenten der Webanwendung, wobei vor allem drei Akteure auf die einzelnen Teilkomponenten zugreifen können. Die Komponente \enquote{accounts} ist für das grundlegenden Account-Management zuständig, sodass ein rollenbasierter Zugriff auf andere Teilkomponenten gewährleistet werden kann. Über die Komponente \enquote{gitload} kann ein Administrator die unter der Komponente \enquote{tasks} gehandhabten Aufgabenstellungen in das System dynamisch integrieren. Lösungen dieser Aufgaben werden in der \enquote{solutions} Komponente behandelt und durch die \enquote{testrunner} Komponente getestet. Als Metaanalysetool steht eine \enquote{statistics} Komponente zur Verfügung. Für die Plagiatsprüfung der Lösungen und die Evaluation der Bonuspunkte ist schließlich die \enquote{grading} Komponente zuständig. Die Teilkomponenten sind teilweise untereinander gekoppelt und greifen auf gemeinsame Ressourcen zu.

\subsection{Zielgruppe}\label{sec:zielgruppe}

Um ein besseres Verständnis für die Interessen, Probleme, Fähigkeiten und Motivationen der Nutzer zu erhalten, soll in der folgenden Sektion auf die Zielgruppe von INLOOP näher eingegangen werden. Die primäre Zielgruppe von INLOOP setzt sich zusammen aus Studierenden verschiedener Studiengänge. Die Studierenden in der Lehrveranstaltung Softwaretechnologie kommen in der Regel aus den Studiengängen der Informatik (Bachelor Informatik, Bachelor Medieninformatik, Diplom Informatik), wo die Lehrveranstaltung regulär im zweiten Semester des jeweiligen Studienablaufplans eingegliedert ist. Einige Studierende kommen jedoch auch aus anderen fachfremden Studiengängen im Rahmen eines Nebenfachs oder aus der Schüleruniversität. Die Studierenden besitzen daher unterschiedliche Vorkenntnisse und Sichtweisen bezüglich der Softwareentwicklung.

\paragraph{Fähigkeiten und Probleme.} Da es sich bei der Plattform um ein fakultatives Ergänzungsangebot zum Lernen von Vorlesungsinhalten handelt, ist den Studierenden auch weitestgehend überlassen, zu welchem Zeitpunkt sie die INLOOP-Aufgaben bearbeiten. Somit kann nicht pauschal davon ausgegangen werden, dass Konzepte aus der Vorlesung zu einem bestimmten Zeitpunkt schon verinnerlicht wurden, auch wenn die Vorstellung in der dazugehörigen Übung oder Vorlesung schon weit zurückliegt. Es stellt sich daher als ein zentrales Problem heraus, die Lerninhalte möglichst angepasst an die unterschiedlichen Lerntypen und Wissensstände anzupassen. Es kommt vor, dass Studierende mit den Aufgaben oder der Fehlersuche überfordert sind und sich hierdurch Frust entwickelt. Die in \Cref{sec:inloop} genannte Unterteilung der Aufgaben nach Schwierigkeitsgrad in Kategorien gibt den Studierenden daher eine metaphorische Leiter, an deren Stufen sie sich Schritt für Schritt zu den komplexesten Aufgaben emporarbeiten können. Analog hierzu sollte schließlich auch bei der Gamification-Erweiterung berücksichtigt werden, dass den Studierenden während der Verwendung zum Teil grundlegende Kompetenzen aus der Softwareentwicklung fehlen. Die auf der grafischen Nutzeroberfläche dargestellten Informationen sollten demnach eine angemessene Komplexität besitzen oder sich hierbei gegebenenfalls graduell an die Fähigkeiten des Nutzers anpassen.

\paragraph{Interessen und Motivationen.} Es obliegt der Entscheidung der Studierenden, ob diese das fakultative Angebot INLOOP nutzen möchten. Die bei der erfolgreichen Lösung der Exam-Aufgaben erreichbaren Bonuspunkte für die Klausur spielen hierbei eine nicht unwesentliche Rolle als eine der zentralen Motivationen. Durch die Bonuspunkte kann die Modulnote bei Bestehen der Klausur signifikant verbessert werden. Außerdem ermöglicht es INLOOP den Studierenden, die im fortgeschrittenen Verlauf der Übung vorgestellten Konzepte, beispielsweise Entwurfsmuster oder UML-Entwurfsdiagramme, nochmals auf die Implementierungsebene zu transkribieren und hierbei iterativ zu lernen. Mit einer ersten Lösung für eine Aufgabe mag der eingereichte Code noch keine der Tests bestehen, aber durch das direkte interaktive Feedback kann der Code sukzessiv modifiziert werden, bis dieser schließlich zum vollständigen Erfolg der Unittests führt. Auf dem Weg hierhin kann der Studierende viele kleine Erfolgserlebnisse erhalten, durch kleine Fortschritte in der Lösungsfindung. Die hierbei oft notwendige Fehlersuche in den Lösungen kann frustrieren, aber durch den Rätselcharakter auch Spaß bereiten und unterstützt in diesem Fall nochmals den Lernprozess. Schließlich ist noch eine weitere Motivation zu beschreiben, welche in der Lösung der Aufgaben liegt, denn hierdurch kann sich der Studierende außerdem selbst kontrollieren und mehr Sicherheit über die eigenen Kompetenzen erhalten.

\subsection{Anwendungsfälle und Akteure}\label{sec:use-cases-actors}

In der folgenden Sektion sollen eine Reihe von Anwendungsfällen konzipiert werden, welche durch die Gamification-Erweiterung erfüllt werden sollen. Diese Anwendungsfälle sollen in den nachfolgenden Kapiteln weiter diskutiert, bei der Konzeption und der Implementation umgesetzt und hierbei weiter verfeinert werden.

\begin{figure}[H]
\centering
\includegraphics[width=\linewidth, bb=0 0 612 340]{use-cases.pdf}
\caption{UML-Use-Case-Diagramm: Anwendungsfälle einer Gamification-Erweiterung für INLOOP in Relation zu den jeweils beteiligten Akteuren.}\label{fig:use-cases}
\end{figure}

\noindent In \Cref{fig:use-cases} ist die bidirektionale Zuordnung zwischen Akteuren und deren Anwendungsfällen zu sehen. Als primäre Akteure wurden der Studierende, der Administrator und das Lehrpersonal aus der Kontextanalyse übernommen. Der Studierende soll flexibel entscheiden können, ob er die Gamification-Erweiterung aktivieren oder deaktivieren möchte (S6). Aktiviert ein Studierender die Gamification-Erweiterung, so soll er die Codequalität seiner eigenen Lösung einsehen können (S1), wobei ein direktes und interaktives Feedback gegeben werden soll (S2). Dieses Feedback soll Vorschläge für die Verbesserung der Codequalität enthalten (S3). Außerdem soll der Studierende einsehen können, ob die eigene Codequalität im Verhältnis zu anderen Studierenden besser oder schlechter ist (S4). Als zentraler Anwendungsfall des Studierenden soll dieser Spaß und Interesse an der Refaktorisierung entdecken können, indem sie für die Refaktorisierung von Code Smells belohnt werden (S5). Aus der Sicht des Administrators soll die Gamification-Erweiterung schnell und bequem aufsetzbar sein (A1). Zusammen mit dem Lehrpersonal teilt sich der Administrator den Anwendungsfall, dass die dem Bewertungsprozess zugrundeliegende Codierungsrichtlinien konfigurierbar sein sollen (AL1). Schließlich besitzt das Lehrpersonal zwei isolierte Anwendungsfälle, genauer die Möglichkeit, häufig auftretende Code Smells (L1) sowie das generelle Interesse an der Refaktorisierung und der Nutzung der Erweiterung (L2) einzusehen. Hieraus soll das Lehrpersonal Rückschlüsse über die Güte der Konfiguration (AL1) und die Nutzung der Erweiterung (S6) schließen können.

\paragraph{Einordnung.} Insgesamt wird durch das Anwendungsfalldiagramm sichtbar, dass vor allem der Studierende als Hauptnutzer der Anwendung Nutzen an der Erweiterung finden soll. Dies reflektiert den Grundgedanken, dass die Gamification-Erweiterung als Hauptziel verfolgt, den Studierenden zur Verbesserung der Codequalität zu motivieren. Die Anwendungsfälle A1, AL1, L1, L2 nehmen hierbei nur eine periphere Rolle ein. Durch die Anwendungsfälle AL1, L1 und L2 wird zusätzlich zu den Anwendungsfällen des Studierenden das Bedürfnis des Lehrpersonals, die Interaktion der Studierenden mit der Erweiterung zu beobachten und zu beeinflussen, berücksichtigt.
