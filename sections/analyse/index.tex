
\chapter{Anforderungsanalyse}\label{ch:requirements}

In diesem Kapitel sollen konkrete Anforderungen an die zu konzipierende Erweiterung ermittelt werden. Hierzu werden zunächst die generellen Rahmenbedingungen beschrieben und anschließend eine konkrete Spezifikation der Anforderungen vorgenommen.

\section{Rahmenbedingungen}\label{sec:rahmenbedingungen}

Da die Konzeption der Erweiterung von INLOOP abhängig ist, soll nachfolgend zunächst dessen Iststand beschrieben, eine Zielgruppenbeschreibung ermittelt sowie konkrete Anwendungsfälle gezeigt werden.

\subsection{Istzustand von INLOOP}

\begin{figure}[H]
\centering
\includegraphics[width=0.5\linewidth, bb=0 0 216 163]{inloop-architecture.pdf}
\caption{Architektur-Metamodell von INLOOP. Entnommen aus \cite{morgenstern_continuous_2018}.}\label{fig:inloop-architecture}
\end{figure}

\noindent INLOOP ist eine auf Python\footnote{Python. \url{https://www.python.org/} (Abgerufen am 15.9.2020)}, Git\footnote{Git. \url{https://git-scm.com/} (Abgerufen am 15.9.2020)}, dem Django Framework\footnote{Django. \url{https://www.djangoproject.com/} (Abgerufen am 15.9.2020)} und Docker\footnote{Docker. \url{https://www.docker.com/} (Abgerufen am 15.9.2020)} basierende Webanwendung. In \Cref{fig:inloop-architecture} ist die äußere Struktur der Anwendung gezeigt. Der Client kann direkt auf die Webanwendung zugreifen, wobei diese mit weiteren Komponenten zur Realisation der einzelnen Funktionalitäten kommuniziert. Zur persistenten Datenspeicherung wird eine Datenbank verwendet. Um asynchrone Prozesse zu planen und auszuführen, wird ein Job Broker verwendet, welcher Hintergrundarbeiter initiiert. Zu diesen asynchronen Aufgaben gehören beispielsweise die kontinuierliche Integration von Aufgaben über ein externes Aufgaben-Repository, aber auch die Steuerung des Aufgaben-Testprozesses in dedizierten containerisierten Laufzeitumgebungen.

\begin{figure}[H]
\centering
\includegraphics[width=\linewidth, bb=0 0 481 333]{inloop-components.pdf}
\caption{UML-Kompositionsstrukturdiagramm: Top-Level-Architektur der Webanwendung INLOOP.}\label{fig:inloop-components}
\end{figure}

\noindent Das Kompositionsstrukturdiagramm in \Cref{fig:inloop-components} zeigt die Komponenten der Webanwendung, wobei vor allem drei Akteure auf die einzelnen Teilkomponenten zugreifen können. Die Komponente \enquote{accounts} ist für das grundlegenden Account-Management zuständig, sodass ein rollenbasierter Zugriff auf andere Teilkomponenten gewährleistet werden kann. Über die Komponente \enquote{gitload} kann ein Administrator die unter der Komponente \enquote{tasks} gehandhabten Aufgabenstellungen in das System dynamisch integrieren. Lösungen dieser Aufgaben werden in der \enquote{solutions} Komponente behandelt und durch die \enquote{testrunner} Komponente getestet. Als Metaanalysetool steht eine \enquote{statistics} Komponente zur Verfügung. Für die Plagiatsprüfung der Lösungen und die Evaluation der Bonuspunkte ist schließlich die \enquote{grading} Komponente zuständig. Die Teilkomponenten sind teilweise untereinander gekoppelt und greifen auf gemeinsame Ressourcen zu.

\subsection{Zielgruppe}\label{sec:zielgruppe}

Um ein besseres Verständnis für die Interessen, Probleme, Fähigkeiten und Motivationen der Nutzer zu erhalten, soll in der folgenden Sektion auf die Zielgruppe von INLOOP näher eingegangen werden. Die primäre Zielgruppe von INLOOP setzt sich zusammen aus Studierenden verschiedener Studiengänge. Die Studierenden in der Lehrveranstaltung Softwaretechnologie kommen in der Regel aus den Studiengängen der Informatik (Bachelor Informatik, Bachelor Medieninformatik, Diplom Informatik), wo die Lehrveranstaltung regulär im zweiten Semester des jeweiligen Studienablaufplans eingegliedert ist. Einige Studierende kommen jedoch auch aus anderen fachfremden Studiengängen im Rahmen eines Nebenfachs oder aus der Schüleruniversität. Die Studierenden besitzen daher unterschiedliche Vorkenntnisse und Sichtweisen bezüglich der Softwareentwicklung.

\paragraph{Fähigkeiten und Probleme.} Da es sich bei der Plattform um ein fakultatives Ergänzungsangebot zum Lernen von Vorlesungsinhalten handelt, ist den Studierenden auch weitestgehend überlassen, zu welchem Zeitpunkt sie die INLOOP-Aufgaben bearbeiten. Somit kann nicht pauschal davon ausgegangen werden, dass Konzepte aus der Vorlesung zu einem bestimmten Zeitpunkt schon verinnerlicht wurden, auch wenn die Vorstellung in der dazugehörigen Übung oder Vorlesung schon weit zurückliegt. Es stellt sich daher als ein zentrales Problem heraus, die Lerninhalte möglichst angepasst an die unterschiedlichen Lerntypen und Wissensstände anzupassen. Es kommt vor, dass Studierende mit den Aufgaben oder der Fehlersuche überfordert sind und sich hierdurch Frust entwickelt. Die in \Cref{sec:inloop} genannte Unterteilung der Aufgaben nach Schwierigkeitsgrad in Kategorien gibt den Studierenden daher eine metaphorische Leiter, an deren Stufen sie sich Schritt für Schritt zu den komplexesten Aufgaben emporarbeiten können. Analog hierzu sollte schließlich auch bei der Gamification-Erweiterung berücksichtigt werden, dass den Studierenden während der Verwendung zum Teil grundlegende Kompetenzen aus der Softwareentwicklung fehlen. Die auf der grafischen Nutzeroberfläche dargestellten Informationen sollten demnach eine angemessene Komplexität besitzen oder sich hierbei gegebenenfalls graduell an die Fähigkeiten des Nutzers anpassen.

\paragraph{Interessen und Motivationen.} Es obliegt der Entscheidung der Studierenden, ob diese das fakultative Angebot INLOOP nutzen möchten. Die bei der erfolgreichen Lösung der Exam-Aufgaben erreichbaren Bonuspunkte für die Klausur spielen hierbei eine nicht unwesentliche Rolle als eine der zentralen Motivationen. Durch die Bonuspunkte kann die Modulnote bei Bestehen der Klausur signifikant verbessert werden. Außerdem ermöglicht es INLOOP den Studierenden, die im fortgeschrittenen Verlauf der Übung vorgestellten Konzepte, beispielsweise Entwurfsmuster oder UML-Entwurfsdiagramme, nochmals auf die Implementierungsebene zu transkribieren und hierbei iterativ zu lernen. Mit einer ersten Lösung für eine Aufgabe mag der eingereichte Code noch keine der Tests bestehen, aber durch das direkte interaktive Feedback kann der Code sukzessiv modifiziert werden, bis dieser schließlich zum vollständigen Erfolg der Unittests führt. Auf dem Weg hierhin kann der Studierende viele kleine Erfolgserlebnisse erhalten, durch kleine Fortschritte in der Lösungsfindung. Die hierbei oft notwendige Fehlersuche in den Lösungen kann frustrieren, aber durch den Rätselcharakter auch Spaß bereiten und unterstützt in diesem Fall nochmals den Lernprozess. Schließlich ist noch eine weitere Motivation zu beschreiben, welche in der Lösung der Aufgaben liegt, denn hierdurch kann sich der Studierende außerdem selbst kontrollieren und mehr Sicherheit über die eigenen Kompetenzen erhalten.

\subsection{Anwendungsfälle und Akteure}\label{sec:use-cases-actors}

In der folgenden Sektion sollen eine Reihe von Anwendungsfällen konzipiert werden, welche durch die Gamification-Erweiterung erfüllt werden sollen. Diese Anwendungsfälle sollen in den nachfolgenden Kapiteln weiter diskutiert, bei der Konzeption und der Implementation umgesetzt und hierbei weiter verfeinert werden.

\begin{figure}[H]
\centering
\includegraphics[width=\linewidth, bb=0 0 612 340]{use-cases.pdf}
\caption{UML-Use-Case-Diagramm: Anwendungsfälle einer Gamification-Erweiterung für INLOOP in Relation zu den jeweils beteiligten Akteuren.}\label{fig:use-cases}
\end{figure}

\noindent In \Cref{fig:use-cases} ist die bidirektionale Zuordnung zwischen Akteuren und deren Anwendungsfällen zu sehen. Als primäre Akteure wurden der Studierende, der Administrator und das Lehrpersonal aus der Kontextanalyse übernommen. Der Studierende soll flexibel entscheiden können, ob er die Gamification-Erweiterung aktivieren oder deaktivieren möchte (S6). Aktiviert ein Studierender die Gamification-Erweiterung, so soll er die Codequalität seiner eigenen Lösung einsehen können (S1), wobei ein direktes und interaktives Feedback gegeben werden soll (S2). Dieses Feedback soll Vorschläge für die Verbesserung der Codequalität enthalten (S3). Außerdem soll der Studierende einsehen können, ob die eigene Codequalität im Verhältnis zu anderen Studierenden besser oder schlechter ist (S4). Als zentraler Anwendungsfall des Studierenden soll dieser Spaß und Interesse an der Refaktorisierung entdecken können, indem sie für die Refaktorisierung von Code Smells belohnt werden (S5). Aus der Sicht des Administrators soll die Gamification-Erweiterung schnell und bequem aufsetzbar sein (A1). Zusammen mit dem Lehrpersonal teilt sich der Administrator den Anwendungsfall, dass die dem Bewertungsprozess zugrundeliegende Codierungsrichtlinien konfigurierbar sein sollen (AL1). Schließlich besitzt das Lehrpersonal zwei isolierte Anwendungsfälle, genauer die Möglichkeit, häufig auftretende Code Smells (L1) sowie das generelle Interesse an der Refaktorisierung und der Nutzung der Erweiterung (L2) einzusehen. Hieraus soll das Lehrpersonal Rückschlüsse über die Güte der Konfiguration (AL1) und die Nutzung der Erweiterung (S6) schließen können.

\paragraph{Einordnung.} Insgesamt wird durch das Anwendungsfalldiagramm sichtbar, dass vor allem der Studierende als Hauptnutzer der Anwendung Nutzen an der Erweiterung finden soll. Dies reflektiert den Grundgedanken, dass die Gamification-Erweiterung als Hauptziel verfolgt, den Studierenden zur Verbesserung der Codequalität zu motivieren. Die Anwendungsfälle A1, AL1, L1, L2 nehmen hierbei nur eine periphere Rolle ein. Durch die Anwendungsfälle AL1, L1 und L2 wird zusätzlich zu den Anwendungsfällen des Studierenden das Bedürfnis des Lehrpersonals, die Interaktion der Studierenden mit der Erweiterung zu beobachten und zu beeinflussen, berücksichtigt.

\section{Spezifikation}\label{sec:spezifikation}

Um die Anforderungen an die Gamification-Erweiterung zu konkretisieren, werden anhand den gezeigten Anwendungsfällen nachfolgend genaue funktionale Anforderungen aufgelistet und konkrete Qualitätsanforderungen beschrieben.

\subsection{Funktionale Anforderungen}\label{sec:funktionale-anforderungen}

In der nachfolgenden Sektion sollen konkrete funktionale Anforderungen organisiert werden, also notwendige Funktionalitäten, welche durch die Gamification-Erweiterung unterstützt werden sollen. Insbesondere die abstrakten Anwendungsfälle und Akteure aus \Cref{sec:use-cases-actors} sollen hierzu als kontextbezogene Grundlage dienen.

\begin{enumerate}
\item Studierende sollen die Möglichkeit haben, die Codequalität der eigenen Lösungen automatisiert auf Grundlage eines Regelwerks für Codierungsrichtlinien analysieren zu lassen.
\item Durch die Analyse von Lösungen soll es für Studierende möglich sein, detektierte Code Smells in den eigenen Lösungen zu erkennen.
\item Die Detektionen sollen für Studierende optisch ansprechend und möglichst leicht verständlich aufbereitet werden.
\item Für die Initialisierung des automatisierten Analyseprozesses und die Auflistung der Detektionen muss die Erweiterung eigene interaktive Komponenten in der grafischen Oberfläche der Gesamtanwendung integrieren und bereitstellen.
\item Die integrierten und bereitgestellten visuellen Komponenten müssen ein direktes und interaktives Feedback ermöglichen.
\item Ein Studierender soll die Möglichkeit haben, die eigene Codequalität mit anderen Studierenden zu vergleichen, sodass die Kompetitivität der Refaktorisierung gefördert wird.
\item Um den Spaß und das Interesse an der Refaktorisierung zu vermitteln, soll der Prozess der Refaktorisierung im Rahmen eines eigenen Game-Designs erlebbar gemacht werden.
\item Dem Studierenden soll es freistehen, ob er die Gamification-Erweiterung nutzen möchte oder nicht.
\item Die Gamification-Erweiterung soll eine hinreichende architekturelle und visuelle Trennung von der Hauptanwendung realisieren, so dass der Aspekt des eigentlichen E-Assessments nicht in den Hintergrund rückt.
\item Die Detektionen der Code Smells sollen mit durch den Studierenden interpretierbaren Lösungsvorschlägen für die Behebung des Code Smells ausgestattet sein und diese auch visuell repräsentieren.
\item Hierzu soll die Erweiterung Beschreibungen zu den im Regelwerk festgesetzten Codierungsrichtlinien bereitstellen.
\item Die der automatisierten Codequalitätsanalyse zugrundeliegenden Regeln sollen, wenn dies auf die jeweilige Regel applizierbar ist, durch einen Administrator oder Lehrpersonal über die Bereitstellung von Konfigurationsdateien konfigurierbar, aktivierbar und deaktivierbar sein.
\item Dem Lehrpersonal soll es ermöglicht werden, eine Übersicht in Form einer Statistik oder in Form einer ähnlichen visuellen Darstellung zu erhalten, welche es ermöglicht, häufig auftretende Code Smells einzusehen.
\item Über eine ähnliche Ansicht soll das Lehrpersonal anhand einer Statistik oder einer ähnlichen visuellen Darstellung einsehen können, wie groß das Interesse der Studierenden an der Refaktorisierung ist.
\item Der Administrator soll die Gamification-Erweiterung möglichst ohne zusätzlichen Aufwand zusammen mit der bestehenden Basisanwendung aufsetzen können.
\end{enumerate}

\subsection{Qualitätsanforderungen}

Analog zu den in \Cref{fig:softwarequalitaet} gezeigten Modellparametern sollen auch für die zu entwickelnde Gamification-Erweiterung bestimmte Qualitätsanforderungen spezifiziert werden. Hierzu werden die übergeordneten Qualitätsattribute zunächst priorisiert.

\begin{table}[H]
\caption{Qualitätsanforderungen an die Gamification-Erweiterung analog zu den in \cite{technical_committee_isoiec_jtc_1sc_7_software_and_systems_engineering_isoiec_2001} gezeigten Modellparametern von interner und externer Softwarequalität. Abkürzungen von links nach rechts: (+) normale Priorität, (++) hohe Priorität, (+++) sehr hohe Priorität, (++++) höchste Priorität.}\label{tab:qualitaetsanforderungen}
\begin{tabular}{|l|c|c|c|c|} \hline
\multicolumn{1}{|l|}{Qualitätsattribut} & \multicolumn{4}{c|}{Priorisierung} \\ \cline{2-5}
                                        & +    & ++   & +++  & ++++ \\ \hline
Funktionalität                          &      &      & x    &      \\ \hline
Verlässlichkeit                         & x    &      &      &      \\ \hline
Nutzbarkeit                             &      &      &      & x    \\ \hline
Effizienz                               &      & x    &      &      \\ \hline
Wartbarkeit                             &      & x    &      &      \\ \hline
Portierbarkeit                          &      & x    &      &      \\ \hline
\end{tabular}
\end{table}

Wie in \Cref{tab:qualitaetsanforderungen} gezeigt, ist die Nutzbarkeit von höchster Priorität. Diese Festlegung beruht auf der Annahme, dass die Erfahrungen des Nutzers, wie in \Cref{sec:gamification-frameworks} und in \Cref{sec:gamification-motivator} beschrieben, eine außerordentlich große Rolle spielen. Außerdem wurde für die Funktionalität eine sehr hohe Priorität angesetzt, worunter nicht nur die Qualität der Erfüllung der funktionalen Anforderungen zählt, sondern auch die Interoperabilität, welche eine größere Rolle bei der Integration in das Basissystem spielt sowie die Sicherheit der Erweiterung, welche insbesondere durch die Speicherung von personenbezogenen Daten und kopierrechtlich geschützten Source-Codes motiviert wird. Bei der Nutzbarkeit spielt auch die Effizienz eine Rolle, weil aus dem Zeitverhalten und der Skalierbarkeit der Gamification-Erweiterung direkte Auswirkungen auf die Interaktivität resultieren können. Ein denkbares Szenario, welches diesen qualitativen Effekt induziert, wäre, dass ein Studierender sehr lange auf die Auswertung der statischen Analyse des eigenen Codes warten muss. Dies würde sich unmittelbar auf die Nutzbarkeit auswirken. Da die Anwendung nicht für die Installation auf einem eingebetteten System mit stark beschränkten Hardwareressourcen gedacht ist, sondern auf einem Webserver mit verhältnismäßig großem Speicherplatz und ausreichender Rechenleistung laufen soll, wurde die Ressourcennutzung der Anwendung mit geringerer Priorität eingeschätzt. Daher wurde insgesamt die Priorität dieses Attributs Effizienz als hoch und nicht als sehr hoch eingestuft. Als weiteres Qualitätsattribut mit hoher Priorität wurde die Wartbarkeit selektiert, insbesondere durch die Relevanz der dahinter liegenden Teilattribute, wie die Änderbarkeit (betont durch Anwendungsfall AL1 in \Cref{fig:use-cases}) sowie die Analysierbarkeit (repräsentiert durch die Anwendungsfälle L1 und L2 in \Cref{fig:use-cases}). Ebenso wurde die Priorität der Portierbarkeit als hoch eingestuft, motiviert durch die Installierbarkeit (gezeigt durch Anwendungsfall A1 in \Cref{fig:use-cases}) und die kontextsensitive Adaptivität, wobei, wie in \Cref{sec:zielgruppe} bereits erwähnt, vor allem der persönliche Kontext und die Fähigkeiten des Nutzers relevant für die Adaptionsmechanismen der Anwendung sind. Weiterhin kontribuiert zur Portierbarkeit auch die Koexistenz der Erweiterung neben anderen Anwendungen, insbesondere INLOOP selbst, wobei dies unmittelbar mit der Interoperabilität zusammenhängt. Ein indikatives Qualitätsszenario hierfür bestünde in der gemeinsamen Nutzung einer Datenbank, sodass die Annahmen und Prozesse der Basisanwendung nicht gestört werden. Die Erweiterung soll entsprechend durch ihre Softwarearchitektur soweit möglich abgekapselt werden. Abschließend, da die Reife der Gamification-Erweiterung im Rahmen der prototypischen Implementation eine untergeordnete Rolle spielt und das durch einen Ausfall des Systems abschätzbare Risiko wegen der dynamischen Ausschaltbarkeit (illustriert durch Anwendungsfall S6 in \Cref{fig:use-cases}) der Funktionalitäten gering ist, wird der Verlässlichkeit eine normale Priorität zugewiesen.


\section{Zusammenfassung}

Im Rahmen einer kurzen Analyse des Istzustands von INLOOP wurden die relevanten Komponenten von INLOOP gezeigt, an welche die Gamification-Erweiterung nachfolgend gekoppelt werden soll. Um außerdem hierauf ein nutzerorientiertes Gamification-Konzept zu erstellen, wurde der Studierende als Hauptnutzer der Anwendung anhand dessen Fähigkeiten, Problemen, Interessen und Motivationen analysiert. Auf Grundlage einer Anwendungsfallanalyse, welche zusätzlich den Administrator und das weitere Lehrpersonal als Akteure inkludiert, wurden konkrete funktionale Anforderungen und Qualitätsanforderungen an die Gamification-Erweiterung spezifiziert.
