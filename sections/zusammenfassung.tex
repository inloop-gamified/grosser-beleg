\chapter{Zusammenfassung}\label{ch:zusammenfassung}

\section{Ergebnisse}\label{sec:ergebnisse-zusammenfassung}

Um die Codequalität der eingereichten Lösungen in INLOOP zu verbessern, wurde eine Erweiterung entworfen, welche INLOOP um ein Gamification-Konzept ergänzt.

\paragraph{\Cref{rq1}} \textit{Welche Codequalitätsmetriken kommen hierfür in Frage und wie können diese im Kontext von INLOOP kombiniert und parametrisiert werden, um signifikante Fehlgestaltungen des Codes zu erkennen, welche die Codequalität der eingereichten Lösungen beeinträchtigen?} \\

\noindent
Bedingt durch die Komplexität und Subjektivität von Codequalität existieren viele verschiedene Metriken, von denen die Konformität des Codes zu bestimmten Codierungsrichtlinien als vielversprechend korrelierend mit der wahrgenommenen Codequalität aus den bisherigen wissenschaftlichen Analysen hervorgegangen ist \cite{rucks_erstellung_2017}\cite{dietz_teaching_2018}. Die Konformitätsmetrik berechnet sich hierbei aus nach deren Schwere gewichteten Detektionen von Fehlgestaltungen (\enquote{Code Smells}). Das Ziel des entworfenen Gamification-Konzeptes ist die Motivation der Nutzer, in eingereichten Lösungen konkrete Code Smells zu identifizieren und zu refaktorisieren, um die Codequalität über die gesteigerte Konformität zu Codierungsrichtlinien und die hiermit zusammenhängenden Qualitätsfaktoren, vor allem die der Lesbarkeit und der Wartbarkeit, zu verbessern \cite{technical_committee_isoiec_jtc_1sc_7_software_and_systems_engineering_isoiec_2001}.


\paragraph{\Cref{rq2}} \textit{Welche Gamification-Elemente sind dafür geeignet, die durch Codequalitätsmetriken erkannten Fehlgestaltungen des Codes an die Nutzer zu kommunizieren, sodass diese motiviert werden, Fehlgestaltungen zu refaktorisieren, im Voraus zu vermeiden und damit die Codequalität der eingereichten Lösungen zu verbessern?} \\

\noindent
Die Gamification-Elemente wurden nach dem Octalysis Gamification-Framework \cite{chou_actionable_2019} im Rahmen eines Game-Designs durch ein nutzerzentriertes Konzept kombiniert. Nutzer können für eingereichte Lösungen eine bestimmte Anzahl an Punkten erreichen, von welcher je nach Schweregrad und Anzahl der aufgetretenen Code Smells Punkte abgezogen werden. Mithilfe von individuellen Code-Smell-Konsultationen wird die Konformität und hierdurch gleichzeitig auch potenzielle Degradationen der Codequalität an Nutzer kommuniziert. Gemeinsam mit progressiven Ranglisten, erreichbaren Erfahrungsstufen, Errungenschaften, Avataren, dem Hinzufügen von Kollegen und dem direkten Feedback über ein Notifikationssystem wurden die gewählten Gamification-Elemente in einem Narrativ integriert und nach dem Octalysis-Framework gewichtet, sodass hierbei zur Ausgeglichenheit sowohl positive/negative, als auch extrinsische und intrinsische Elemente vertreten sind. Hierbei wurden Game-Design-Elemente und Konzepte verwandter Arbeiten, insbesondere der Arbeit von Händler und Neumann \cite{haendler_serious_2019} sowie Dos Santos et al. \cite{dos_santos_cleangame_2019} an verschiedenen Stellen wiederverwendet, beispielsweise in Form von interaktiven Fragen zu Lösungen anderer Nutzer. Im Fokus des nutzerzentrierten Ansatzes stand bei der Konzeption stets die Motivation des Nutzers, sich über potenzielle oder tatsächliche Degradationen der Codequalität seiner Lösungen und den Lösungen anderer zu informieren, und selbstständig Wissen zu akquirieren und schließlich auf die eigenen Lösungen anzuwenden.


\paragraph{\Cref{rq3}} \textit{Wie können die Codequalitätsmetriken mit den Gamification-Elementen in einer Gamification-Erweiterung kombiniert werden?} \\

\noindent
Als technische Grundlage der Gamification-Erweiterung dient eine didaktisch durch Erklärungen gestützte semantische Code Smell Detektion im Zusammenspiel mit der statischen Codeanalyse mithilfe von im akademischen Kontext entwickelten Regelwerken für Codierungsrichtlinien \cite{dietz_teaching_2018}. Auf Verstöße gegen diese Richtlinien wird beim Einreichen einer Aufgabe automatisiert geprüft. Die hierbei entstandenen Detektionen werden in einem Violation-Modell vereinheitlicht und durch textuelle Erklärungen ergänzt, um diese schließlich in individuellen Code-Smell-Konsultationen sinnvoll darstellen zu können.

\paragraph{\Cref{rq4}} \textit{Welche Auswirkungen hat die Einführung der Gamification-Erweiterung auf die Motivation, die Codequalität eingereichter Lösungen zu verbessern?} \\

\noindent
Mithilfe einer zweikomponentigen Evaluation wurde das Gamification-Konzept anhand einer prototypischen Implementation auf dessen Motivationswirkung untersucht. Durch eine Expertenevaluation nach Prinzipien des Discount Usability Engineering konnte gezeigt werden, dass die Benutzung des Prototypen ein intuitives Interesse am Entdecken, Lernen und zur Refaktorisierung von Code Smells anregen kann. Die erhobenen Umfragedaten der Studierendenevaluation unterstützen diese Hypothese insgesamt, insbesondere zusätzlich auch die Sinnhaftigkeit und Notwendigkeit der Ergänzung von Erklärungen zu den detektierten Code Smells, um die Hintergründe der individuellen Detektionen besser nachvollziehen zu können.

\section{Ausblick}\label{sec:ausblick-zusammenfassung}

Die erhobenen Evaluationsergebnisse suggerieren, dass eine Finalisierung des Prototypen sowie dessen Produktionseinführung zur Integration der Codequalität zum fakultativ orientierte didaktische Konzept von INLOOP beitragen und die Codequalität der eingereichten Lösungen signifikant verbessern kann. Hierzu wurden in der durchgeführten Expertenevaluation weitere Verbesserungsmöglichkeiten der Gamification-Erweiterung identifiziert, beispielsweise eine Ergänzung der semantischen Code-Smell-Detektion um konkrete Codebeispiele. Eine Studie über die Effekte und die Nutzung der sich in Produktion befindlichen Gamification-Erweiterung wäre denkbar und könnte zum wissenschaftlichen Verständnis der didaktischen Wirksamkeit von Gamification in der universitären Lehre mit Bezug auf Codequalität beitragen, zu der bisher nur wenige Konzepte und kaum Evidenz vorliegen \cite{dos_santos_cleangame_2019}. Außerdem könnte die Gamification-Erweiterung auf andere Bereiche der Softwarequalität ausgedehnt werden, beispielsweise auf die Motivation von besonders performantem Code.

Ein weiteres Problem stellt die für diese Arbeit grundlegende Detektion von Code Smells über statische Codeanalysetools dar, welche durch eine verhältnismäßig hohe falsch-positive Detektionsrate geprägt ist \cite{yang_expert_2019}. Yang et al. schildern, 35 bis 91 Prozent der Detektionen in Codebeispielen würden von erfahrenen Entwicklern als \enquote{nicht zu behandeln} eingestuft werden. Die stichprobenartige Ermittlung von Code Smells zur Generierung von Umfragen im Rahmen der Studierendenevaluation zeigt, dass die Sensitivität der Detektion in INLOOP-Lösungen bereits bei der Einbeziehung von nur einem statischen Codeanalysetool (in diesem Fall Checkstyle) recht hoch ist. Um die Genauigkeit zu verbessern, bietet sich damit eine geeignete Filterung nach bestimmten Heuristiken an, welche falsch-positive Detektionen erkennen. Ein etablierter Ansatz hierfür ist die Einstufung der Detektionen nach bestimmten statistischen Ranking-Systemen \cite{engler_bugs_2001}\cite{allier_framework_2012}. Yang et al. schlagen als Lösung des Problems einen Machine-Learning-Algorithmus zur binären Klassifikation der Detektionen in die Kategorien \enquote{unberechtigt} und \enquote{berechtigt} vor, wobei der Algorithmus zunächst automatisiert auf Änderungshistorien von bekannten Bibliotheken vortrainiert und danach durch einen Experten iterativ mit Klassifikationen angepasst wird \cite{yang_expert_2019}. Anhand dieser Erkenntnisse könnte ein durch Machine Learning getriebenes Ranking-Framework für die Code-Smell-Detektion weiter dazu beitragen, Code Smells kontextsensitiv vorauszuwählen und somit die Präzision der statischen Codeanalyse zu verbessern.
