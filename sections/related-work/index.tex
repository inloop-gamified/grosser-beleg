\chapter{Related Work}\label{ch:related-work}

Im folgenden Kapitel werden verschiedene verwandte Ansätze aus der Domäne der Gamification diskutiert, deren direktes oder indirektes Ziel die Verbesserung der Codequalität im universitären Kontext ist. Einen allgemeinen Überblick über die Gamification im Software Engineering zeigen die systematischen Studien von Pedreira et al. \cite{pedreira_gamification_2014} und von Alhammad und Moreno \cite{alhammad_gamification_2018}, Letztere mit dem für diese Arbeit relevanten didaktischem Fokus. Obwohl Alhammad und Moreno anhand der systematischen Analyse Gamification als vielversprechendes Prinzip für die Verbesserung der Lehre der Softwareentwicklung einschätzen, gibt es zur Applikation von Gamification in diesem Kontext nur wenig empirische Forschung \cite{alhammad_gamification_2018}. Campolina et al. analysierten die Gründe hinter der geringen Verbreitung von Gamification in diesem Kontext \cite{campolina_games_2018}. Außerdem lässt sich beobachten, dass einige der Ansätze, Gamification in der Lehre der Softwareentwicklung einzusetzen, andere Schwerpunkte als die Codequalität behandeln, beispielsweise das Requirements Engineering \cite{mei_teng_game-based_2018}\cite{campolina_games_2018}, den Scrum-Prozess \cite{neto_case_2019}, die Integration von Peer-Review-Konzepten \cite{indriasari_gamification_2020}\cite{berkling_presenting_2019}, die Verbesserung von hochintensiven Programmierkursen \cite{akpolat_enhancing_2014} oder die Motivation von Softwaretests \cite{rojas_code_2016}\cite{sheth_halo_2011}. Nennenswert ist auch der eher unkonventionelle Ansatz von Baars und Meester, die ein Java-Refaktorisierungsspiel in Minecraft integrierten \cite{baars_codearena_2019}. Elezi et al. zeigen eine Anwendung von Gamification zur Motivation der Refaktorisierung von Code \cite{elezi_game_2016}, allerdings nicht in einem akademischen Kontext, sondern mit praxiserfahrenen Softwareentwicklern.

Dos Santos et al. kamen daher 2019 nach deren Einschätzung zu dem Schluss, dass ihre Arbeit die erste wissenschaftliche Kontribution im Bereich der Motivation zur Refaktorisierung von Code Smells durch Gamification im akademischen Kontext ist \cite{dos_santos_cleangame_2019}. Nachfolgend wird zunächst das von dos Santos et al. als \enquote{CleanGames} bezeichnete Konzept diskutiert. Anschließend wird ein verwandter Ansatz von Händler und Neumann beschrieben, der die Motivation der Refaktorisierung von Code über Serious Games behandelt \cite{haendler_serious_2019}. Zur Erkennung von Refaktorisierungskandidaten eignet sich die statische Codeanalyse. Mit Fokus auf diese beschäftigen sich Dietz et al. mit der Frage, wie Codequalität gelehrt werden kann, um schließlich eine Grundlage für die automatisierte Codequalitätsanalyse in einem universitären Kontext zu konzeptionieren \cite{dietz_teaching_2018}. Dieser Ansatz wird in der abschließenden Sektion diskutiert.


\section{CleanGame: Gamifying the Identification of Code Smells}\label{sec:cleangame}

Wie in \Cref{sec:refaktorisierung} bereits näher beschrieben, bietet die Refaktorisierung eine Möglichkeit, die Softwarequalität zu verbessern, indem interne Komponenten abgeändert werden, während die externe Funktionalität unverändert bleibt. Dieser Prozess inkludiert eine systematische Suche und Behandlung von Refaktorisierungskandidaten. In welchem Maße eine Refaktorisierung durchgeführt werden kann und wie erfolgreich diese der Softwarequalität zugutekommt, hängt somit auch von der Menge und der Sinnhaftigkeit der gefundenen Refaktorisierungskandidaten ab. Auf Grundlage dieser Prämisse untersuchten dos Santos et al., wie die Suche von Refaktorisierungskandidaten in einem akademischen Kontext durch Gamification spielerisch motiviert werden kann und welche quantitativen Auswirkungen dies auf die Fähigkeit der Nutzer, Refaktorisierungskandidaten zu finden, haben kann \cite{dos_santos_cleangame_2019}.

\subsection{Game-Design}

Die von den Forschern hierzu entworfene Webanwendung trägt den Namen \enquote{CleanGame} und besteht aus einem Quiz über Code Smells und einem Spiel, bei dem Refaktorisierungskandidaten in Codeschnipseln gefunden werden sollen. Über das Quiz sollen die Kernkonzepte der Refaktorisierung reflektiert werden, während durch das andere Modul die Suche nach Refaktorisierungskandidaten trainiert werden soll. Das Softwaretool lässt sich in die Kategorie der Serious Games einordnen.

\subsubsection{Code-Smell-Quiz}\label{sec:code-smell-quiz}

Im Code-Smell-Quiz werden dem Nutzer Fragen zu ausgewählten Code-Smells gestellt. Die Forscher fokussieren sich hierbei auf die folgenden fünf konkreten Code-Smells, basierend auf einer Metaanalyse von Code Smells im akademischen und industriellen Kontext von Sharma und Spinellis \cite{sharma_survey_2017}:

\begin{enumerate}
\item \textbf{Large Class}: Hierunter fallen Klassen, welche zu viele Verantwortlichkeiten tragen. Dieses Anti-Pattern ist auch bekannt unter den Namen \enquote{Blob} oder \enquote{Gott-Klasse} \cite{sourcemakingcom_design_2020}.
\item \textbf{Long Method}: Dies bezeichnet Methoden, welche zu viele Zeilen von Code beinhalten.
\item \textbf{Divergent Change}: Hiermit beschreiben die Autoren eine Klasse, die sich im Verlauf der Entwicklung häufig und aus verschiedenen Gründen ändert.
\item \textbf{Feature Envy}: Unter diesem Namen werden Klassen kategorisiert, welche deren Funktion hauptsächlich durch die Kommunikation mit anderen Klassen realisieren, also kaum eigene Funktionalität bereitstellen. Häufig steht dieses Anti-Pattern in Verbindung mit der Auslagerung von Daten-Klassen \cite{refactoringguru_data_2020}.
\item \textbf{Shotgun Surgery}: Durch diese Bezeichnung beschreiben dos Santos et al. Modifikationen in Klassen, welche viele kleine Änderungen in anderen Klassen hervorrufen.
\end{enumerate}

\noindent Innerhalb des Quiz werden dem Nutzer Fragen zu diesen fünf Kategorien gestellt, wobei die Antwortmöglichkeiten festgelegt sind. Um den Frageprozess zu illustrieren, zeigen dos Santos et al. folgende Frage inklusive deren Antwortmöglichkeiten:

\begin{quote}
Um den Shotgun Surgery Code Smell zu beseitigen, können Refaktorisierungen verwendet werden. Welche der folgenden Refaktorisierungen würden Sie nutzen, um dies zu realisieren?

\begin{itemize}
\item[A] Vererbung mit einer Delegation ersetzen
\item[B] Inline Class
\item[C] Methoden extrahieren
\item[D] Klasse extrahieren
\end{itemize}
\cite[sinngemäß übersetzt. S. 4, Abb. 1]{dos_santos_cleangame_2019}
\end{quote}

\noindent Um diese Frage, ohne zu raten, erfolgreich zu lösen, muss der Nutzer ein komplexes, vernetztes Fachwissen über Code Smells und Refaktorisierung besitzen. Zunächst muss der Nutzer wissen, worum es sich bei der \enquote{Shotgun Surgery} handelt und, dass das zugrundeliegende Problem dieses Anti-Patterns die Zersplitterung einer Verantwortlichkeit über verschiedene Klassen ist \cite{refactoringguru_shotgun_2020}. Eine Möglichkeit der Refaktorisierung wäre die Extraktion von zur Verantwortlichkeit gehörenden Methoden aus den verschiedenen Klassen und deren Reintegration in eine zentrale Klasse. Somit wäre Antwortmöglichkeit C bereits richtig. Durch die Migration der Methoden können jedoch auch Klassen entstehen, welche selbst keine Verantwortlichkeiten mehr besitzen. Diese Klassen werden als \enquote{Inline Class} \cite{refactoringguru_inline_2020}, meist auch als \enquote{Data Class} bezeichnet und fallen unter die Kategorie \enquote{Feature Envy}. Richtig wäre also auch Antwortmöglichkeit B, um die möglichen transitiven Konsequenzen aus der initialen Refaktorisierung zu behandeln. Dieses Beispiel illustriert, wie über den Ansatz der Autoren über ein Quiz komplexes vernetztes Fachwissen zur Refaktorisierung abgefragt, reflektiert und erlernt werden kann.

\subsubsection{Verwendete Gamification-Elemente}

Während des Quiz erhält der Nutzer Punkte für das richtige Beantworten einer Frage. Für das falsche Beantworten einer Frage bekommt der Nutzer Punkte abgezogen. Noch während des Quiz kann der Nutzer den durch seinen aktuellen Punktestand errechneten Rang in der Rangliste einsehen. Für eine Frage hat der Nutzer nur begrenzt Zeit zur Verfügung. Fragen können übersprungen werden, während Schnelligkeit bei deren Beantwortung durch einen Zeitbonus belohnt wird. Beantwortet der Nutzer einige aufeinander folgende Fragen richtig, so erhält er auch hierfür Bonuspunkte. Weiterhin sieht der Nutzer seinen eigenen Fortschritt über einen Fortschrittsbalken in der Webanwendung. Der Nutzer hat, nach \cite[S. 4, Abb. 1]{dos_santos_cleangame_2019} außerdem die Möglichkeit, sich zu registrieren und für den erstellten Account einen rudimentären Avatar anzulegen.

\subsubsection{Code-Smell-Suche}

Bei der zweiten Kernkomponente von CleanGame handelt es sich um eine Code-Smell-Suche, bei der dem Nutzer Code-Schnipsel gezeigt werden, zu denen er den passenden Code Smell aus einer festen Auswahl von Code Smells selektieren soll. Die Code-Smell-Suche ähnelt dem Quiz in deren Aufbau und Funktionsweise, denn auch hier erhält der Nutzer Punkte für richtige Antworten und bekommt Punkte für falsche Antworten abgezogen. Dies wird erweitert durch die Fähigkeit des Nutzers, die Punkte als Währung gegen Tipps einzutauschen. Ein Tipp kann die Angabe von Regeln sein, die mit dem Code Smell in Verbindung stehen, oder eine Einschränkung des zu betrachtenden Codeabschnittes, oder die dem Code Smell zugrundeliegende Definition.

\subsection{Ergebnisse}

Dos Santos et al. prüften ihr Konzept im akademischen Kontext anhand einer quantitativen Analyse über die Messung der durch die Probanden gefundenen Code Smells und einer qualitativen Analyse der persönlichen Einstellung der Nutzer gegenüber dem Game-Design. Bei der quantitativen Analyse zweier randomisierter Nutzergruppen, wobei die eine Nutzergruppe CleanGame und die andere Nutzergruppe eine IDE im Voraus zum Training der Identifikation von Code Smells nutzte, konnten die Nutzer der CleanGame-Gruppe nach dem Training doppelt so viele Code Smells identifizieren, wie Nutzer der IDE-Gruppe. Für die Selektion der Trainingsbeispiele und der Evaluationsbeispiele nutzten dos Santos et al. validierte Code Smells der Plattform \enquote{Landfill} \cite{palomba_landfill_2015}. In einer qualitativen Analyse über eine Umfrage ermittelten dos Santos et al., dass die Attitüde der Probanden gegenüber dem evaluierten Serious Game überwiegend positiv war. Als verbesserungswürdig schätzten Probanden unter anderem vor allem Probleme mit der Benutzeroberfläche ein, aber auch, dass kein direktes Feedback über die richtige Antwort einer Frage gegeben wurde und, dass die Punkte eines jeweiligen Nutzers öffentlich einsehbar waren. Als positiv schätzen die Probanden wiederum unter anderem vor allem die spielerische und interaktive Auseinandersetzung mit dem Thema ein sowie die Unterstützung beim Verstehen von Code Smells und die Kompetitivität. Die qualitative Evaluation der ästhetischen Wahrnehmung des Tools ergab weiterhin, dass durch den Ansatz der Autoren die Relevanz, die Lernwahrnehmung und die soziale Interaktion besonders betont würde.

\noindent Die Autoren interpretieren diese Ergebnisse als Unterstützung der Hypothese, dass Gamification in diesem Kontext geeignet ist. Zu berücksichtigen ist jedoch, dass sowohl die qualitative als auch die quantitative Analyse nur mit 18 Probanden durchgeführt wurden und somit nur bedingt aussagekräftig sind. Daher merken dos Santos et al. an, dass die Ergebnisse weiterhin validiert werden müssen, um eine hinreichende Konfidenz über die Wirksamkeit des Effekts der Gamification im universitären Kontext zur Motivation von Refaktorisierung zu erhalten.

\section{Serious Refactoring Games}\label{table:bloom-taxonomy-serious}

Die Refaktorisierung von Code mit Hinblick auf die Verbesserung der Codequalität ist ein komplexer Prozess. Ein zu dem vorgestellten \enquote{CleanGame} Konzept ähnliches Serious Game konzipierten Händler und Neumann. Die Autoren betrachten die didaktische Komplexität der Refaktorisierung von Code anhand der Taxonomie von Bloom \cite{bloom_taxonomy_1956}\cite{anderson_taxonomy_2000}, welche Denkfähigkeiten geordnet nach der mentalen Beanspruchung, von \enquote{lower order thinking skills} wie dem Merken von Informationen bis hin zu \enquote{higher order thinking skills} wie der Evaluation und der Kreation kategorisiert. Händler und Neumann kommen dabei zu dem Ergebnis, dass die Refaktorisierung von Code alle der Kategorien in Blooms Taxonomie beansprucht \cite{haendler_serious_2019}. Diese Aufschlüsselung ist in \Cref{tab:bloom} gezeigt.

\begin{table}[H]
\caption{Kognitive Fähigkeiten bei der Refaktorisierung anhand von Blooms Taxonomie. Sinngemäß aus der tabellarische Darstellung in \cite[S. 3]{haendler_serious_2019} abstrahiert.}\label{tab:bloom}
\begin{tabularx}{\textwidth}{{
    >{\hsize=.5\hsize\linewidth=\hsize}X
    >{\hsize=1.5\hsize\linewidth=\hsize}X
}}
\hline
Stufe in Blooms Taxonomie & Beispiel für eine benötigte Fähigkeit bei der Refaktorisierung \\ \hline \hline
1. Wissen & Die Regeln und Schritte für die Planung und die Durchführung der Refaktorisierung kennen. \\ \hline
2. Verstehen & Die Regeln und Schritte für die Planung und die Durchführung der Refaktorisierung verstehen. \\ \hline
3. Applikation & Die Regeln und Schritte für die Planung und die Durchführung der Refaktorisierung anwenden. \\ \hline
4. Analyse & Den Quellcode eines Systems analysieren und hieraus Kandidaten für die Refaktorisierung ableiten. \\ \hline
5. Bewertung & Mehrere Möglichkeiten der Refaktorisierung von Refaktorisierungskandidaten miteinander vergleichen und sich für die bestmögliche Lösung entscheiden. \\ \hline
6. Kreation & Entwicklung und Verbesserung von Systemen und Arbeitsweisen, welche die Refaktorisierung von Softwaresystemen vereinfachen oder strukturieren. \\
\hline
\end{tabularx}
\end{table}

Anhand dieser Erkenntnis kritisieren die Autoren, dass bisherige didaktische Lösungen für die Lehre des Refactorings nur künstliche, kleine Aufgaben beinhalten, die aber nicht ausreichend seien, um tiefergehende Kompetenzen wie analytisches Denken und die Bewertung von Lösungen zu vermitteln. Stattdessen schlagen sie ein Spiel vor, bei dem dem Nutzer ein größeres, in dessen Kontext funktionales, Codefragment präsentiert wird und es daraufhin des Nutzers Aufgabe ist, die in diesem Codefragment vorhandenen Code Smells zu beseitigen, ohne die von außen erkennbare Funktionalität abzuändern. Hierbei kann der Nutzer gegen einen Codequalitätsbenchmark (Technische Schulden) oder gegen echte Mitspieler antreten. In diesem Kontext beschreiben Händler und Neumann eine technische Grundlage für das Game-Design solcher \enquote{Serious Refactoring Games} auf Basis von existierenden Codeanalyse-Tools (Test-Frameworks, Softwarequalitätsmetriken) und Regressionstests.

Die vorgeschlagene Lösung von Händler und Neumann hat das Ziel, insbesondere auch die höherliegenden Stufen 3-6 der Taxonomie von Bloom (\Cref{table:bloom-taxonomy-serious}) didaktisch zu vermitteln, da sich existierende Lösungen eher auf die darunter liegenden Stufen orientierten, analysieren die Autoren. Durch die Konfrontation mit Codefragmenten aus \enquote{echten} Anwendungen werde eine praxisnähere Situation simuliert, bei der Nutzer die Artefakte zunächst analysieren müssen und danach Alternativen evaluieren müssen, um die Code Smells zu beseitigen, wodurch höhere Lernziele in Blooms Taxonomie repräsentiert würden.

\subsection{Game-Design}

Der Spieler erhält den Quellcode der zu refaktorisierenden Datei(en). Des Spielers Aufgabe ist es hiernach, den Quellcode auf mögliche zu refaktorisierende Stellen (Refaktorisierungskandidaten) zu untersuchen und diese auszuwählen. Wenn der Spieler einen oder mehrere Kandidaten ausgewählt hat, muss er Modifikation(en) des Quellcodes planen und ausführen, die den jeweiligen ausgewählten Kandidaten behandelt. Signalisiert der Spieler, dass dieser die Modifikation(en) abgeschlossen hat, wird die editierte Datei im ursprünglichen System integriert und auf Funktionalität geprüft (Regressionstest). Nun kann es zwei Ergebnisse dieser Interaktion geben:

\begin{itemize}
\item Der Programmcode ist valid und alle Regressionstests waren erfolgreich. Infolgedessen wird die Qualität des refaktorisierten Codes analysiert und ein Score für die technischen Schulden ermittelt.
\item Einer oder mehrere Regressionstest schlagen fehl oder der Programmcode ist anderweitig invalid. Die Qualität des refaktorisierten Codes wird nicht analysiert und es wird kein Score für die technischen Schulden ermittelt.
\end{itemize}

\noindent Das ermittelte Ergebnis wird dem Spieler präsentiert und dieser erhält, wenn vorhanden, den zum refaktorisierten Code berechneten Score. Der Spieler erhält nun die Option, weitere Modifikationen an den bereits ausgewählten Kandidaten vorzunehmen oder nach weiteren Kandidaten zu suchen. Das Spiel ist nach dem beschriebenen Ablauf in Spielzüge unterteilt, welche nach deren Abschluss als inkorrekt, korrekt oder erfolgreich gelten können:

\begin{itemize}
\item Inkorrekt: der Quellcode ist invalid und konnte nicht auf Funktionalität oder Qualität geprüft werden.
\item Korrekt: der Quellcode ist valid, die technischen Schulden haben sich aber im Vergleich zum vorigen Spielzug nicht verringert.
\item Erfolgreich: der Quellcode ist valid und die technischen Schulden haben sich im Vergleich zum vorigen Spielzug verringert.
\end{itemize}

\noindent Händler und Neumann schlagen mehrere erweiterte Spielmodi vor, darunter ein Einzelspielermodus, wobei der Spieler gegen die Uhr spielt sowie mehrere Mehrspielermodi (parallel vs. alternierend, kompetitiv vs. kollaborativ).

\subsubsection{Ziele eines Serious Refactoring Games}

Das oberste Ziel des Spielers besteht darin, das Codefragment auf eine solche Weise zu refaktorisieren, sodass die Funktionsfähigkeit der Software weiterhin durch die Regressionstests verifiziert werden kann, sich in diesem Kontext also nach außen nicht verändert, während die technischen Schulden minimiert werden.

\subsubsection{Technische Realisation}

Das von Händler und Neumann konzipierte Softwaresystem ist in die folgenden Komponenten unterteilt:

\begin{itemize}
\item Die \textbf{Game Control} koordiniert die Interaktion von Spielern und dem Testframework bzw. der Qualitätsanalyse.
\item Das \textbf{Test Framework} führt Regressionstests auf der eingereichten Software aus.
\item Der \textbf{Quality Analyzer} analysiert die Qualität des Codefragments.
\item Die \textbf{GUI Component} kapselt das visuelle Feedback und die gesamte visuelle Interaktion.
\end{itemize}

\noindent Als Grundlage schlagen Händler und Neumann eine Messung der internen technischen Qualität über die Akkumulation eines TD-Scores (\Cref{begriff:td-score}) vor.

\subsection{Ergebnisse}

Händler und Neumann planen die Evaluation des vorgestellten Konzeptes im universitären Kontext im Rahmen zukünftiger Arbeit \cite[S. 9]{haendler_serious_2019}. Zu dem vorgestellten Konzept der Serious Refactoring Games wurden noch keine experimentellen Resultate publiziert (Stand: 30. Juni 2020), so auch nicht in der in \cite{haendler_interactive_2019} von Händler et al. auf Grundlage dessen fortgeführten Arbeit, welche sich im Kontrast zu den vorgestellten Serious Refactoring Games auf die Realisation der Interaktivität über ein direktes Feedback im Rahmen eines Tutoring-Systems fokussiert.

\subsection{Framework}\label{sec:haendler-framework}

In \cite{haendler_framework_2019} diskutieren Händler und Neumann die Serious Refactoring Games und das oben genannte Tutoring-System erneut und konzipieren anhand dessen ein Framework, welches die Lernziele in Blooms Taxonomie möglichst gut abbildet. Hierbei soll das vorgestellte Serious Refactoring Game durch folgende Konzepte aus dem Tutoring-System komplementiert werden:

\begin{itemize}
\item Die Bereitstellung von direktem, interaktiven Feedback durch die Präsentation der Resultate von Regressionstests.
\item Der interaktive Vergleich des durch den Code zu reflektierenden Referenz-UML-Diagramms (Soll-Zustand) mit einem aus dem eingereichten Code ermittelten UML-Diagramm (Ist-Zustand).
\end{itemize}

\noindent Ähnlich zu den in \Cref{sec:cleangame} beschriebenen CleanGames soll hierdurch ein weiterer Spielmodus umgesetzt werden, dessen Ziel die Eliminierung von Code Smells aus einem gegebenen Codefragment ist, während die Regressionstests auf eine fortexistierende Funktionalität und damit die Validität der Refaktorisierung prüfen. Als eine weitere Möglichkeit zur Ergänzung des Frameworks nennen Händler und Neumann die Integration eines Quiz über Refaktorisierung, ähnlich dessen in CleanGames, welches in \Cref{sec:cleangame} näher beschrieben wurde.


\section{Teaching Clean Code}

Dietz et al. befassten sich mit den Forschungsfragen, wie die Programmierung von \enquote{Clean Code}, also Code mit hoher Konformität zu Codierungsrichtlinien, in einem universitären Kontext gelehrt werden kann und wie dies mit einer zunehmenden Anzahl an Studierenden skalierbar umsetzbar ist. Dazu gehen die Autoren auf Probleme ein, die in der Lehre von Clean Code im universitären Kontext bestehen. Über die Integration eines didaktischen Konzeptes auf Grundlage von persönlichem Feedback sammelten Dietz et al. Erkenntnisse über die Strukturierung und Auswahl von Codierungsrichtlinien. Mithilfe dieser empirischen Auswahl wird schließlich eine Grundlage für deren Integration in einem E-Learning-System wie ArTEMiS \cite{krusche_artemis_2018}, einem INLOOP ähnelnden E-Learning-System für Softwaretechnologie der Technischen Universität München, mit Fokus auf der Verbesserung der Codequalität gegeben.

\subsubsection{Herausforderungen in der universitären Lehre}

Dietz et al. beschreiben, dass Studierende der Informatik häufig aufgrund der Lehrstruktur des Studiengangs nicht ausreichend motiviert würden, sich über die praktische Applikation von theoretischen Konzepten hinweg autodidaktisch mit der qualitativen Verbesserung des eigenen Codes zu befassen, weil dieser oft nach der Entwicklung vergessen oder \enquote{weggeworfen} würde. Motiviert hierdurch analysierten Dietz et al., wie die Einhaltung von Codierungsrichtlinien in einem universitären Kontext, ähnlich dem dieser Arbeit, gelehrt werden kann. So integrierten sie ein didaktisches Konzept in den Präsenzbetrieb einer Lehrveranstaltung mit den folgenden drei Kernelementen:

\begin{itemize}
\item Interaktive Diskussion von Refaktorisierungskandidaten in der Vorlesung zwischen den Studierenden und dem Vorlesenden anhand von Lösungen zu Aufgaben
\item Detaillierte Code Reviews von Lösungen der Übungsgruppen durch Lehrpersonal
\item Integration einer Aufgabenstellung in die mündliche Prüfung, bei der Refaktorisierungskandidaten in einem unbekannten Code-Schnipsel erkannt werden sollen
\end{itemize}

\noindent Dieses didaktische Konzept erprobten die Autoren an der Universität Bamberg mehrere Jahre und evaluierten schließlich mithilfe von statischer Codeanalyse die messbaren Änderungen in der Konformität des eingereichten Codes zu einem festen Regelwerk. Sie stellten als Tendenz fest, dass sich die allgemeine Struktur des Codes verbesserte, während andere Regelverstöße zunahmen, wobei die Autoren diese Zunahmen teilweise der Art der Aufgabe attribuieren \cite[S. 3]{dietz_teaching_2018}.

Bei diesem Konzept spielen die personellen Ressourcen eine große Rolle, so dass bei der Umsetzung jedes der drei Kernelemente Lehrpersonal vonnöten ist. Mit einer steigenden Teilnehmerzahl in den Informatik-Studiengängen steigt damit auch der Bedarf an den personellen Ressourcen zur Umsetzung eines solchen Konzeptes, wodurch der Bedarf einer skalierbareren Lösung entsteht.

\subsubsection{Auswahl von Codierungsrichtlinien}\label{sec:qualityreview}

Um die Skalierbarkeit des Konzepts zu verbessern, schlagen Dietz et al. vor, die Erfahrungen aus der Umsetzung des Konzepts in eine Wissensbasis zu überführen \cite{dietz_teaching_2018} und auf Grundlage dieser eine automatisierte statische Codeanalyse durchzuführen. Die Autoren verfassten die Wissensbasis als Buch unter dem Titel \enquote{Java by Comparison: Become a Java Craftsman in 70 Examples.} mit den am weitesten verbreiteten Problemen inklusive Codebeispielen und Erklärungen \cite{harrer_java_2018}.

Anhand der universitären Wissensbasis erstellten die Autoren ein Meta-Tool, welches in Summe etwas mehr als 100 Regeln der Codeanalyseframeworks Checkstyle, PMD und FindBugs/SpotBugs kombiniert und genutzt werden kann, um Codierungsrichtlinien, welche sich für einen universitären Kontext eignen, automatisiert in E-Learning-Systeme zu integrieren. Bei der Zusammenstellung der Richtlinien verfolgen Dietz et al. die Maxime, diese möglichst sinnvoll für alle Lerntypen auszuwählen und die Anzahl der falsch-positiven Detektionen gering zu halten. Bestimmte Programmierstile sollen darüber hinaus nicht forciert werden. Das entwickelte Meta-Tool ist unter dem Namen QualityReview auf GitHub\footnote{QualityReview. \url{https://github.com/LinusDietz/QualityReview} (Abgerufen am 13.6.2020)} verfügbar und beinhaltet Regelsätze für die einzelnen statischen Analysetools.


\section{Zusammenfassung}

Die selbstständige Refaktorisierung von Code mit Hinblick auf die Verbesserung der Codequalität scheint in den praktischen Teilen der Informatik-Studiengängen oft unterrepräsentiert oder nicht hinreichend motiviert zu sein. Dietz et al. zeigten, wie \enquote{Clean Code} als didaktisches Konzept in die Präsenzlehre integriert werden kann \cite{dietz_teaching_2018}. Die mit steigenden Studierendenzahlen an Relevanz gewinnenden E-Learning-Umgebungen für Softwareentwicklung können koexistierend zur Präsenzlehre jedoch weitere Wege erschließen, Studierende zu motivieren, nicht nur funktionell einwandfreien, sondern auch hochqualitativen Code zu schreiben. Um dies skalierbar zu erreichen, können Studierende motiviert werden, ausgewählte Codierungsrichtlinien einzuhalten. Eine sorgfältige Auswahl von solchen Richtlinien, für einen universitären Kontext mit Studierenden aus frühen Semestern, entwickelten Dietz et al. über mehrere Jahre iterativ zu den Erfahrungen aus der Integration von Codequalität als didaktisches Konzept in die universitäre Lehre. Das entwickelte Regelwerk bietet eine gute Grundlage für eine im akademischen Kontext geeignete Detektion von Code Smells und damit auch eine gute Grundlage für die zu entwickelnde, hierauf basierende Gamification-Erweiterung.

Zur Motivation der Refaktorisierung als Mittel der qualitativen Verbesserung von Code im universitären Kontext wurden die Serious-Game-Konzepte CleanGame \cite{dos_santos_cleangame_2019} und Serious Refactoring Games \cite{haendler_serious_2019} diskutiert. Händler und Neumann zeigen im Rahmen der Serious Refactoring Games, wie höhergestellte Lernziele nach Blooms Taxonomie bedient werden können. Dos Santos et al. zeigen, wie komplexe Lerninhalte über ein zweikomponentiges Quiz vermittelt werden können. Die hieraus extrahierbaren Teilkonzepte können für die zu erstellende Gamification-Erweiterung verwendet werden, um höherwertige Lernziele abzudecken. Beide Arbeiten diskutieren hierzu ein eigenes Game-Design, welches jedoch nur in der Arbeit zu CleanGame durch dos Santos et al. experimentell evaluiert wird. Dos Santos et al. schildern, dass es weiterer Forschung bedarf, um konfidente Aussagen über die Wirksamkeit von Gamification im universitären Kontext als Motivator zur Verbesserung der Codequalität treffen zu können.

Sowohl dos Santos et al. als auch Händler und Neumann und Dietz et al. zeigen Ansätze, wie Studierenden ein besseres Verständnis von Codequalität vermittelt werden kann und wie sie motiviert werden können, dieses Verständnis im Rahmen der Refaktorisierung praktisch anzuwenden. Dos Santos et al. sowie Händler und Neumann fokussierten sich konzentrisch hierauf, wobei deren Arbeiten hierfür ein eigenständiges Serious-Game-Konzept entwickelten. Die vorgestellten Serious Games können, ähnlich zu INLOOP, als fakultatives Angebot neben einer Lehrveranstaltung für die Motivation der Refaktorisierung von Code in Form von Einzelanwendungen eingesetzt werden. Im Unterschied zu den Serious Games jedoch werden in INLOOP darüber hinaus gehende Lehrinhalte vermittelt, zum Beispiel die Interpretation von UML-Modellen. Die in den folgenden Kapiteln zu konzipierende Gamification-Erweiterung unterscheidet sich somit vor allem in dem nachfolgenden Punkt von den vorgestellten Arbeiten. Der primäre Fokus auf dem Erlernen von objektorientierten Konzepten anhand der Programmiersprache Java soll in INLOOP nicht durch das Erlernen von Codequalitätsrichtlinien verdrängt werden. Vielmehr muss sich die zu konzipierende Gamification-Erweiterung nahtlos in das Gesamtsystem von INLOOP integrieren und dieses subtil genug ergänzen, um im Lernprozess nicht als hinderlich wahrgenommen zu werden. Die von Händler und Neumann im Rahmen von \cite{haendler_interactive_2019}, \cite{haendler_serious_2019} und schließlich \cite{haendler_framework_2019} systematisch vorgestellten Konzepte dienen dennoch zusammen mit der Arbeit von dos Santos et al. \cite{dos_santos_cleangame_2019} als gute Grundlage für die Analyse, Konzeption und Implementation der Gamification-Erweiterung in den folgenden Kapiteln.

