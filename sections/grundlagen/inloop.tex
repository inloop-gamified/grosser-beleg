\section{INLOOP}\label{sec:inloop}

INLOOP ist eine Webanwendung, die an der Technischen Universität Dresden entwickelt wurde. Die Webanwendung wird im Rahmen der Lehrveranstaltung Softwaretechnologie genutzt, um deren didaktisches Konzept durch ein fakultatives Angebot von Online-Programmieraufgaben zu erweitern. Das didaktische Konzept beinhaltet als Lernziele dabei unter anderem, dass die Studierenden anhand der Programmiersprache Java objektorientierte Konzepte (Entwurfsmuster, Klassenbibliotheken, UML-Modelle) implementieren und diese Implementation einer Software-Qualitätssicherung unterziehen können \cite{asmann_modulbeschreibung_2010}. In INLOOP können Programmieraufgaben veröffentlicht und in verschiedene Kategorien unterteilt werden. Die Kategorien (Basic, Lesson, Exam) sind hierbei nach steigender Komplexität und Schwierigkeit geordnet. Die in diese Kategorien eingegliederten Programmieraufgaben bestehen aus einer textuellen Beschreibung. In der textuellen Beschreibung der meisten Aufgaben sind außerdem Diagramme integriert, welche zum Beispiel die Klassenstruktur der zu implementierenden Software repräsentieren (in Form von UML-Analyse/Entwurf-Klassendiagrammen) oder die Reihenfolge und Art von zwischen Komponenten der Software ausgetauschten Informationen (in Form von UML-Sequenzdiagrammen oder UML-Zustandsdiagrammen) aufzeigen. Aufgaben können von Nutzern der Plattform, entweder in einer eigenen Entwicklungsumgebung oder in einem integrierten Online-Editor, von zuhause bearbeitet werden. Die eingereichten Lösungen werden in der Plattform auf funktionelle Korrektheit geprüft und hierdurch automatisiert ausgewertet. Zum Schluss eines jeden Bearbeitungsprozesses wird ein direktes Feedback zur funktionellen Korrektheit der Teilkomponenten der eingereichten Lösung präsentiert. Besteht eine Lösung nicht alle funktionellen Tests, so wird sie als \enquote{nicht bestanden} gewertet. Als logische Konsequenz ist eine Lösung auch nur dann \enquote{bestanden}, wenn alle funktionellen Tests erfolgreich waren. INLOOP gliedert sich mit diesem didaktisch orientierten Grundkonzept in die so genannten E-Learning-Systeme ein und hat somit den primären Zweck, Lehrinhalte zu vermitteln und den Nutzern der Plattform beim Erlernen neuer Inhalte zu helfen. Da die Inhalte darüber hinaus auch abgefragt und bewertet werden können, repräsentiert INLOOP gleichzeitig ein so genanntes E-Assessment-System \cite{handke_e-learning_2012}. \Cref{fig:assessment} zeigt den im Fokus stehenden Lernprozess als Iterationszyklus, zu dessen Schluss das durch die automatisierte Beurteilung in INLOOP realisierte summative Assessment steht.

\begin{figure}[H]
\centering
\includegraphics[width=\linewidth, bb=0 0 462 132]{assessment.pdf}
\caption{Varianten des Assessments anhand des iterativen Lernzyklus. Angelehnt an \cite[S. 44]{handke_e-learning_2012} und basierend auf \cite[S. 41]{crisp_e-assessment_2007}.}\label{fig:assessment}
\end{figure}

\noindent Die Aufgaben in INLOOP werden über das laufende Semester für die Studierenden in zeitlich aufeinanderfolgenden Abschnitten über ein eigenes Versionsmanagement freigegeben, sodass zu Anfang des Semesters noch keine der schwierigeren und komplexeren Exam-Aufgaben verfügbar sind. Die Studierenden können sich somit zu Beginn des Semesters auf die verhältnismäßig einfacheren und grundlegenderen Aufgaben konzentrieren. Hierbei ist zu beachten, dass Aufgaben als befristet definiert werden können, also nur bis zum Verfall einer bestimmten Deadline abgegeben werden können. Zur Motivation der Studierenden, das fakultative Angebot zu nutzen, werden als Anreiz Bonuspunkte für die Klausur verliehen. Die genauen Details und Rahmenbedingungen können die Studierenden dabei auf INLOOP einsehen\footnote{Bonuspoint rules. \url{https://inloop.inf.tu-dresden.de/about/bonuspoint-rules/} (Abgerufen am 13.6.2020)}. Bonuspunkte werden hierbei nur vergeben, wenn die in Frage kommende Lösung nicht plagiiert wurde. Für die Plagiatsprüfung der Java-basierten Lösungen wird JPlag\footnote{JPlag. \url{https://jplag.ipd.kit.edu/} (Abgerufen am 13.6.2020)} verwendet. Mit der Einführung von INLOOP und mithilfe der Bonuspunkte als Motivator konnte beobachtet werden, dass sich hierdurch vermehrt Studierende an den bereitgestellten Programmieraufgaben probierten und schließlich im Vergleich besser in der Abschlussprüfung der Lehrveranstaltung abschnitten \cite{morgenstern_continuous_2018}.
